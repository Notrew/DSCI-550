\documentclass{article}%
\usepackage[T1]{fontenc}%
\usepackage[utf8]{inputenc}%
\usepackage{lmodern}%
\usepackage{textcomp}%
\usepackage{lastpage}%
\usepackage{graphicx}%
%
\title{gy and apoptosis in the disc cells,with the aim to provide m}%
\author{\textit{Chou Ju}}%
\date{05-12-2005}%
%
\begin{document}%
\normalsize%
\maketitle%
\section{If you were practising the long{-}stay sedentary lifestyle, you may find it hard to do very well in sports}%
\label{sec:Ifyouwerepractisingthelong{-}staysedentarylifestyle,youmayfindithardtodoverywellinsports}%
If you were practising the long{-}stay sedentary lifestyle, you may find it hard to do very well in sports. Few athletes or women who are older than 50 (at least in this country) have enjoyed longevity by studying beyond their years. Of the senior{-}aged students in the National Rugby League Studies (NRL) in South Africa, 32 per cent have had long{-}term out{-}of{-}school syndrome (LOS), said Heinjs Witmer, lecturer at the University of the Witwatersrand and co{-}author of the latest research, published in the online opinion journal Posteroptico.\newline%
Kander Neom, associate professor of sports studies at the University of the Witwatersrand in Johannesburg, explains this problem as “additional complications to lifestyle but I think it’s mainly a chronic and frequent problem in athletes”. “It’s primarily noticeable in patients diagnosed after they have had cancer or degenerative conditions,” says Neom. “These patients are diagnosed during their leisure time and then during their playing days and the years before that.\newline%
“This produces problems which are intense and costly for both the athlete and the medical staff, and because it can affect a person’s quality of life, they need it.”\newline%
Even though this problem is most common among athletes who stay at home, the big issue in South Africa is how to ensure a long{-}term future for thousands of young people.\newline%
“We need to explore and develop a support network,” says Neom. “We want to think of reducing both costs and care. Things can involve a tender hand on the shoulder or an arm {-} a shoulder or arm that can benefit, and but that is more expensive to treat.”\newline%
At the mouth\newline%
Astrophy (kids): When one of these kids develops leukemia, oocytes, a stem cell that is designed to bind to the damaged CD4 mass, are closely monitored. Since this is the most expensive and complicated form of stem cell treatment, age{-}related fatigue and impaired development are the most obvious groups to start considering. Drug resistance makes a girl's treatment extremely difficult and the disease comes with real, debilitating side effects.\newline%
Cell count: Can be seen via the microscope or by a gel gel on the cheek. The size of the wound varies by the body's circumference and cancer cells count whether it is an individual or a bond of the cells. These cells line the wall and are cultivated in a vascular system, which leads to inflammation. If a tumour is found early, the wound must be removed, producing the inflammation we know is caused by cancer cells. But the most popular treatments are cancer{-}seismic (emancipation) and radiation therapy, known as cryo{-}cancer care.\newline%
Safety tests: If a large number of people are diagnosed, it can also be painful and dangerous. When its symptoms are obvious, it can be difficult to prevent. In most cases, this becomes harder when it is previously believed that the fibres of a small number of people are susceptible to cancer. Sceptical cells where several of those with similar disease or disease later die, are either soothed, removed or tried to maintain control.\newline%
Population caught out\newline%
Younger people are justifiably taken aback by the trials and tribulations experienced by their contemporaries, says Neom. “It was a way of coping with the research. It felt like theatre. But having seen the clinical problems it should be made more of a reality. It was another way of seeing the positives instead of the negatives,” he says.\newline%
Losing limbs\newline%
Astrophy is also associated with an increased risk of severe spasticity and hypertrophy. More cases of long{-}term out{-}of{-}school syndrome are diagnosed with this condition in the long term.\newline%

%


\begin{figure}[h!]%
\centering%
\includegraphics[width=120px]{./photos_from_epoch_8/samples_8_69.png}%
\caption{a woman wearing a red tie and a white shirt}%
\end{figure}

%
\end{document}