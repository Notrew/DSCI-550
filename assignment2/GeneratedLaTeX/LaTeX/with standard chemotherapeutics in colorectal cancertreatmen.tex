\documentclass{article}%
\usepackage[T1]{fontenc}%
\usepackage[utf8]{inputenc}%
\usepackage{lmodern}%
\usepackage{textcomp}%
\usepackage{lastpage}%
\usepackage{graphicx}%
%
\title{with standard chemotherapeutics in colorectal cancertreatmen}%
\author{\textit{Yang Ye}}%
\date{10-29-1999}%
%
\begin{document}%
\normalsize%
\maketitle%
\section{Stressing that the use of this vital molecular platform has not harmed patients with colorectal cancer, another organisation, the Our Helps Counts, yesterday opened a dental clinic in Gauteng}%
\label{sec:Stressingthattheuseofthisvitalmolecularplatformhasnotharmedpatientswithcolorectalcancer,anotherorganisation,theOurHelpsCounts,yesterdayopenedadentalclinicinGauteng}%
Stressing that the use of this vital molecular platform has not harmed patients with colorectal cancer, another organisation, the Our Helps Counts, yesterday opened a dental clinic in Gauteng.\newline%
Dr Edgar S. Ibe, CEO, Our Helps Counts, had explained in an earlier submission to the Uganda Children's Parliament that the new dental clinic would be an extension of another NGO, Medicina Labs. The new clinic in Gauteng, courtesy of the organization based in Africa, would hold the 11 months examination of a 300 total patient population and therefore provide treatment for those who need chemotherapy for colorectal cancer.\newline%
For example, the "Salary Rate" for the CAT scan required for someone in the same stage of the disease would have to increase by 20\% to cut the cancer{-}screening costs for some patients. This would have to be a one{-}off payment.\newline%
Medina is headquartered in South Africa where its activities were still going strong after high profile arrest in Kenya last December. The dental clinic in Uganda would also function as a collaboration between Medina and The Children's Hospital of Namibia.\newline%
Currently Medina is the medical centre with the highest operating cost in Africa. Medina, Africa's largest independent dental service provider, has been in line with its commitment to building dental clinics in villages for the last 60 years. This facility is part of what the organisation calls the "Ministry of Khumalo Listed hospitals" programme to provide dental surgery to households.\newline%
Once all the dental surgeries are completed, individuals will receive universal access to the same dental services as people living in rural areas. This has been the case before, when even the illiterate in rural areas refused the procedure.\newline%
Although most children in rural areas lack access to dental, the burden of the problem is still felt by the children. To address this, Medina has launched a two{-}year programme with local private business partnerships to provide a hundred other educational, and voluntary, dental services to children, the elderly, pregnant women, and the less privileged.\newline%
In the hope of alleviating this burden, Medina has also invested \$5000 in marketing and distributing the dental service, and this all culminated into the signing of a joint cultural partnerships agreement with Kampala Protocol School of Tropical Medicine, further driving that vision forward.\newline%
Although the dental services are still totally non{-}hospitals, Medina is now central to the ministry's efforts to help save lives. A European adoption expert for children with DISA of Uganda, Eva Kalouka{-}Zohoringa, said in an interview that the adoption authority could save as many as 400 children from the epidemic and the new clinic would help to give girls the confidence to continue as young women to serve in the diaspora.\newline%
There were always doubts whether to extend the dental service to a whole country, but this response from the long established Ministry of Women and Children with Development programmes is significant.\newline%

%


\begin{figure}[h!]%
\centering%
\includegraphics[width=120px]{./photos_from_epoch_8/samples_8_442.png}%
\caption{a man in a suit and tie holding a teddy bear}%
\end{figure}

%
\end{document}