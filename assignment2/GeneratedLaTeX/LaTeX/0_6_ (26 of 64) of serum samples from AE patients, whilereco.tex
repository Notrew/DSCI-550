\documentclass{article}%
\usepackage[T1]{fontenc}%
\usepackage[utf8]{inputenc}%
\usepackage{lmodern}%
\usepackage{textcomp}%
\usepackage{lastpage}%
\usepackage{graphicx}%
%
\title{0\_6\_ (26 of 64) of serum samples from AE patients, whilereco}%
\author{\textit{Yen Wang}}%
\date{02-21-2005}%
%
\begin{document}%
\normalsize%
\maketitle%
\section{Symptoms of a serious kidney failure\newline%
Kidney failure occurs when a clot in blood clots in arteries that have previously blocked essential heart functions}%
\label{sec:SymptomsofaseriouskidneyfailureKidneyfailureoccurswhenaclotinbloodclotsinarteriesthathavepreviouslyblockedessentialheartfunctions}%
Symptoms of a serious kidney failure\newline%
Kidney failure occurs when a clot in blood clots in arteries that have previously blocked essential heart functions. Patients with a history of renal failure or the possibility of having untreated hypertension suffer either condition. Life expectancy at diagnosis has increased by 10 years. Outcomes are much improved by improving medication administered in check of renal symptoms in patients with kidney failure, as compared to drugs in prescription. In particular, medications that address antiseptic effects are now recommended when there is a serious kidney failure.\newline%
This contrasts with the last three years, in which fewer than 1 percent of patients with renal failure or the possibility of having untreated hypertension suffered their condition. This continued improvement provides excellent support to other key patient groups.\newline%
Committed to\newline%
The study established a consensus on 12 subtypes of kidney failure {-} renal failure, renal failure, renal defect, renal condition, renal cell injury (RCI) and renal disability (RID). Consistent with the recommendations of these studies, the study used instruments and evidence{-}based instrumentation to observe over 12 years of renal impairment without treatment of patients with a history of renal failure or the possibility of having untreated hypertension. The team found that the infection rate with renal defect decreased by 42 per cent when comparing patients who developed the infection with patients who did not develop the infection. The complications associated with the infection remained the same in patients with renal defect, with the 10{-}year window reduced by 48 per cent, while those with renal defect and kidney injury worsened their rates from 47 per cent to 45 per cent.\newline%
Every two months, OMS presents comprehensive plans and a detailed patient{-}treatment assessment to assess the effectiveness of therapy in patients with renal disease.\newline%
Key summary\newline%
The report concludes with the following points about the efficacy of different renal medications:\newline%
1. The assessment requires a detailed and highly advance training for a (pronounced y{-}ha{-}kath'get) public affairs office (2. patient instruction from NHS; so long as patients are treated in a timely manner) and a practical knowledge of both existing controls, advice and health systems.\newline%
2. Patients can take the hospital hand{-}in{-}hand programme at home or deliver the drugs at a nursing home.\newline%

%


\begin{figure}[h!]%
\centering%
\includegraphics[width=120px]{./photos_from_epoch_8/samples_8_215.png}%
\caption{a man with a beard and a red tie .}%
\end{figure}

%
\end{document}