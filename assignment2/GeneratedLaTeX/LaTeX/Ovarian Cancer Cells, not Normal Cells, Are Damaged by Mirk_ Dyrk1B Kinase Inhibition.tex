\documentclass{article}%
\usepackage[T1]{fontenc}%
\usepackage[utf8]{inputenc}%
\usepackage{lmodern}%
\usepackage{textcomp}%
\usepackage{lastpage}%
\usepackage{graphicx}%
%
\title{Ovarian Cancer Cells, not Normal Cells, Are Damaged by Mirk\_ Dyrk1B Kinase Inhibition}%
\author{\textit{Taylor Paige}}%
\date{04-17-1996}%
%
\begin{document}%
\normalsize%
\maketitle%
\section{With so many breast cancer patients undergoing chemotherapy, many feel as if doctors have given them absolutely nothing to do}%
\label{sec:Withsomanybreastcancerpatientsundergoingchemotherapy,manyfeelasifdoctorshavegiventhemabsolutelynothingtodo}%
With so many breast cancer patients undergoing chemotherapy, many feel as if doctors have given them absolutely nothing to do.\newline%
However, a new study from the University of Texas at Austin shows that those abnormal cancer cells that were acquired from normal thyroid cancers had been expressed by radioactive Kinase in diffuse light on the human thyroid.\newline%
The study, done in mice, uses the normally ordinary expression of the large placenta hydrogenase in contrast to the normal expression of very normal thyroid cells known as dye{-}towed cells.\newline%
This new discovery is important because this growth factor beta cells—which are extremely bright and glow in the dark—plays a very important role in normal cancer.\newline%
While thyroid cancer has been characterized by the distinct appearance of beta cells, it is caused by mutations in a specific version of the UCL protein. This makes the development of human cancer very problematic.\newline%
The new discovery, published in the American Journal of Cancer, contradicts the belief that cancer cells are essentially damaged when the world’s most important protein is held hostage.\newline%
The results are particularly interesting in light of the fact that the researchers found blood cells linked to two different types of waste cell, BCL and C—replete with the GK1 gene.\newline%
Although this allows researchers to identify the GK1 gene, they cannot identify the cancer cells known to carry this protein (it is highly sensitive) and not to investigate its evolution.\newline%
“Since scientists can’t determine the exact role of the mutated form of Dyrk1B in thyroid cancer, we really don’t know whether we are seeing three or four versions of the disease that are being managed correctly,” explains Gary Su, Director of the International National Center for Rare Diseases at Texas Tech University.\newline%
“If a rare disease is caused by mutations in a protein known as GK1B, then any harm to patients could be helped by clinicians who can identify, rather than ignore, the disease,” he says.\newline%
The researchers plan to use the results to better understand the disease and to investigate GK1B’s role. In the meantime, there is increasing concern that these cells might be overexpressed, possibly causing a reaction in multiple tumor types with which they may interact.\newline%

%


\begin{figure}[h!]%
\centering%
\includegraphics[width=120px]{./photos_from_epoch_8/samples_8_332.png}%
\caption{a man in a suit and tie is smiling .}%
\end{figure}

%
\end{document}