\documentclass{article}%
\usepackage[T1]{fontenc}%
\usepackage[utf8]{inputenc}%
\usepackage{lmodern}%
\usepackage{textcomp}%
\usepackage{lastpage}%
\usepackage{graphicx}%
%
\title{\_ invasion and migration\_ AKT*Corresponding Authors\_ Natalia}%
\author{\textit{Yüan Xiu}}%
\date{09-25-2004}%
%
\begin{document}%
\normalsize%
\maketitle%
\section{2009 is another chapter in this super event where you can just try to convince your audience to remember how much they love … {[}{]}\newline%
This year’s documentary feature ‘Out of Iraq’ features Natalia, their journalist sister Aziz to talk about two similar upcoming events, covering the financial crisis in Iraq and the US invasion of Afghanistan}%
\label{sec:2009isanotherchapterinthissupereventwhereyoucanjusttrytoconvinceyouraudiencetorememberhowmuchtheyloveThisyearsdocumentaryfeatureOutofIraqfeaturesNatalia,theirjournalistsisterAziztotalkabouttwosimilarupcomingevents,coveringthefinancialcrisisinIraqandtheUSinvasionofAfghanistan}%
2009 is another chapter in this super event where you can just try to convince your audience to remember how much they love … {[}{]}\newline%
This year’s documentary feature ‘Out of Iraq’ features Natalia, their journalist sister Aziz to talk about two similar upcoming events, covering the financial crisis in Iraq and the US invasion of Afghanistan.\newline%
One concerns the economic situation affecting 3.5 million people in South Africa who have lost their jobs and are in fear that their pension (currently £2500 per year) will shrink by the day.\newline%
The other runs counter to the 4.5 million South Africans who are bombarded by government media when it comes to their health.\newline%
Announcing the documentary, Natalia says: “We are concerned by all aspects of the post{-}war South African economy. But we also want to remind South Africans of what happens when democracy is put into question and, in reality, you win elections without the power of a state.”\newline%
She adds that Pakistan was “the most intolerant of extremism” with its brutal security laws and draconian police and judiciary rules, however ironically, her film says: “Unlike those in western cities, they still suffer from thugs the likes of which we have never seen before.”\newline%
Stephen, the project’s director, adds: “Natalia is not a genius about her subject matter. But she is a frank voice about working with leaders in struggling cities.”\newline%
“It can be hard for a woman to explain her identity, but there’s no mistaking her for an intelligent girl such as Aziz.\newline%
“Working with her was tricky because she came off as a single woman and I got her to admit to the complexity of her political circumstances. “And yet now we have a cause, a story.”\newline%
Opposite Aziz, Douglas, a Labour MP for Jericho, believes the Government could do more to control “reckless marketisation” which is threatening to destabilise South Africa.\newline%
The 44{-}year{-}old says: “It seems like a really backwards approach to try to stop competition.”\newline%
After releasing the documentary on the same date as the police crackdown on black marketeers in Beijing, Douglas has taken the lead in pressing the government to be more proactive in implementing “progressive” policies such as the Youth Industries Free Enterprise Project.\newline%
He says: “I am working with the government to look at the effects of quantitative easing on youths from all ethnic backgrounds. I don’t know of anybody who is bold enough to implement a plan like this.”\newline%
Kirin co{-}producer Janak, who has also worked with the South African Broadcasting Corporation, agrees: “The film raises important questions, but it is a lot more important to consider since they are incredibly connected to the economy.”\newline%
Among those asked to explain their films are Ian Knose, who foresees a “dirty war” on South Africa and claims the 2008 election is going to be a referendum on “the divide between South Africa and Pakistan”, and Zambian musician Jason Allan, who says his film has received more attention than even a documentary by Amnesty International.\newline%
Another Arrick has been scrutinising South Africa since the filming of the film ‘Operation Shock and Awe’ in May.\newline%
He admits: “The film was coldly elided because it was very, very dark.”\newline%
On the one hand it’s “funny” to see some journalists quoting election allegories, but then there’s the issue of whether or not to “meet next Monday” as due to the civil unrest.\newline%
“It’s very moving, but this film is not about elections, it’s about how this urban and rural sub{-}continent is sewn together. It’s not about putting down fight,” he says.\newline%

%


\begin{figure}[h!]%
\centering%
\includegraphics[width=120px]{./photos_from_epoch_8/samples_8_197.png}%
\caption{a woman in a dress shirt and a tie .}%
\end{figure}

%
\end{document}