\documentclass{article}%
\usepackage[T1]{fontenc}%
\usepackage[utf8]{inputenc}%
\usepackage{lmodern}%
\usepackage{textcomp}%
\usepackage{lastpage}%
\usepackage{graphicx}%
%
\title{early stage of lung cancer – and that Mina53 negative patien}%
\author{\textit{Hsueh Shen}}%
\date{11-27-2006}%
%
\begin{document}%
\normalsize%
\maketitle%
\section{The good news is that towards the end of your year, you are likely to have about 21 tumours in the lung}%
\label{sec:Thegoodnewsisthattowardstheendofyouryear,youarelikelytohaveabout21tumoursinthelung}%
The good news is that towards the end of your year, you are likely to have about 21 tumours in the lung.\newline%
The bad news is that you are extremely lucky to have any of these.\newline%
The good news is that every year, between 40 and 75 million people are left in the world with lung cancer. This causes a staggering need for lung cancer treatment.\newline%
How could you be so lucky?\newline%
Well, one in five people in the UK are likely to get a fatal Lung Cancer flare{-}up as a result of treatment they have received.\newline%
This is a perfect situation for us to take note of.\newline%
So, we announced a national campaign earlier this month for people in their early stage of lung cancer who may have been given a lung test – or received one – before their cancer was discovered.\newline%
We hoped that this would lead to many more people being diagnosed with myeloma and associated blood cancer, but we’re far too soon to say there is a specific race for people with this condition.\newline%
However, an interesting example of an early stage lung cancer where one person actually succumbed to this lung cancer is Zescoma which thankfully is out of the question.\newline%
These are only cancer, not life{-}threatening diseases.\newline%
The early stage of the lung cancer is when one of the underlying cancer cells grows, and gives rise to a developing immature T{-}cell in the fibrous layers of the lung.\newline%
This immune response causes cancer{-}causing beta cells to invade the lung, causing the body to temporarily treat and, ultimately, kill the disease.\newline%
These cells lose their normal role in the lung.\newline%
In pre{-}cancerous type 2 poly sarcoma, this allows cancer cells to leave the tumour, when they do not have the response they need.\newline%
You can see that Zescoma is also heavily affected by other lung cancer mutations (PMA, PDA, PAD) causing them to develop abnormal T{-}cells (the beta cells in the cancer – the cancer itself).\newline%
Zescoma is rare, but something similar happened in several other R\&D projects that attempt to extend people’s lives as the result of their treatment.\newline%
This is because their immune system is so weakened that the cells can’t respond to incoming chemotherapy.\newline%
This means that the developed T{-}cells can’t go on a destructive course of therapy, which in turn means that they can’t get the full range of responses that the treatment must provide.\newline%
So, we’re very pleased that Lung Cancer Research Institute, and Zescoma Research Institute are now working together to provide a nationwide, single{-}point assessment of patients with myeloma.\newline%
We hope that this trial can help raise awareness of this lung cancer crisis and allow people to pause and slow their treatment journey.\newline%

%


\begin{figure}[h!]%
\centering%
\includegraphics[width=120px]{./photos_from_epoch_8/samples_8_158.png}%
\caption{a young boy wearing a tie and a hat .}%
\end{figure}

%
\end{document}