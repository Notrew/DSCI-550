\documentclass{article}%
\usepackage[T1]{fontenc}%
\usepackage[utf8]{inputenc}%
\usepackage{lmodern}%
\usepackage{textcomp}%
\usepackage{lastpage}%
\usepackage{graphicx}%
%
\title{\_ To date, several studies have been conducted for the ident}%
\author{\textit{Kang Bo}}%
\date{11-07-2005}%
%
\begin{document}%
\normalsize%
\maketitle%
\section{The Billie Billie Foundation announced this morning that the Global Mental Health Initiative, famous for its long{-}running campaign to prevent mental health in young children and young adults (as in their families, and “Over 50”), has had an enrolment in its partnership initiative from 189 countries}%
\label{sec:TheBillieBillieFoundationannouncedthismorningthattheGlobalMentalHealthInitiative,famousforitslong{-}runningcampaigntopreventmentalhealthinyoungchildrenandyoungadults(asintheirfamilies,andOver50),hashadanenrolmentinitspartnershipinitiativefrom189countries}%
The Billie Billie Foundation announced this morning that the Global Mental Health Initiative, famous for its long{-}running campaign to prevent mental health in young children and young adults (as in their families, and “Over 50”), has had an enrolment in its partnership initiative from 189 countries.\newline%
But new science, according to the Billie Billie Foundation, which supports the universal principle that we all have a role to play, may allow further research into the wide range of causes and consequences of mental health conditions.\newline%
“The Billie Billie Foundation is proud to provide research funded by the Billie Billie Foundation into how the campaign to prevent mental health among young people can improve people’s lives,” founder and CEO, Elizabeth Dove said.\newline%
While the campaign has received considerable support from academics, environmentalists and other pro{-}mental health campaigners, Dove stresses there is a significant lag between the effectiveness of the campaign and health outcomes.\newline%
To help safeguard future generations, Dove suggests that longitudinal epidemiological studies may be the best intervention for preventing mental health in children.\newline%
“Research is important in preventing mental health in young people, so providing evidence on the effectiveness of interventions in prevention and symptom management can be in large part of this research. While personal experience is not always in the best hands, it is important that younger generations benefit from these studies.”\newline%

%


\begin{figure}[h!]%
\centering%
\includegraphics[width=120px]{./photos_from_epoch_8/samples_8_234.png}%
\caption{a man in a baseball uniform holding a bat .}%
\end{figure}

%
\end{document}