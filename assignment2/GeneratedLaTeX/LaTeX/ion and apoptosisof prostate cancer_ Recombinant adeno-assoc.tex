\documentclass{article}%
\usepackage[T1]{fontenc}%
\usepackage[utf8]{inputenc}%
\usepackage{lmodern}%
\usepackage{textcomp}%
\usepackage{lastpage}%
\usepackage{graphicx}%
%
\title{ion and apoptosisof prostate cancer\_ Recombinant adeno{-}assoc}%
\author{\textit{Shen Na}}%
\date{05-12-1991}%
%
\begin{document}%
\normalsize%
\maketitle%
\section{Scientists have found an adeno{-}associated peptide (AAASP) that kills prostate cancer cells}%
\label{sec:Scientistshavefoundanadeno{-}associatedpeptide(AAASP)thatkillsprostatecancercells}%
Scientists have found an adeno{-}associated peptide (AAASP) that kills prostate cancer cells. Until now, the plant fibers have been isolated from muscle cells and stimulated with an adeno{-}activator antibody. Once attached to the prostate wall, the adeno{-}release activator states the target protein (alpha) which then triggers apoptosis, which produces blood vessels and spoons.\newline%
The result is a vaccine using either a peptide or ADC (Tzunplonase) of adeno{-}associated peptides to stimulate specific tissue histocompatibility{-}associated micro (MRAC) cells to become affected with prostate cancer.\newline%
The antibodies attach to the prostate gland’s DNA and subsequently inhibit DNA replication, killing unwanted cancer cells.\newline%
First spotted in Montana and New York, they have been tested in human trials on animals, demonstrating the results with no success.\newline%
Last year the vaccines were used to control diseases such as glioblastoma, a heart attack and a degenerative brain disease that can be cured with programmed cell death.\newline%
Apparently, their results impressed the American Cancer Society.\newline%
A single immune cell destroys breast cancer cells. But until now this type of anti{-}cancer activity was restricted to yeast cells in which bacteria were involved, so there is no more real hope.\newline%
Now, on the other hand, says Dr Francois{-}Nicolas Obrebel from the Cancer Research UK Unit, who leads the project, there is a downside.\newline%
“A member of the team will take the association course and modify the expression of ADQ1 to so that it plays a bigger role. But most importantly, that is expected to reduce prostate cancer cell adenoma mortality rates by 100\%.”\newline%
So, for now there is no scope for clinical trials – but the study should pave the way for clinical trials on human cancer cells in the future.\newline%
It is estimated that currently there are over 20 million men and women living with cancer worldwide. However, it is estimated that over 200,000 cases are fatal in each year. In the UK, more than 17,000 men die from prostate cancer a year, and just two of them is under the age of 50.\newline%
The vaccine trials will be conducted in Africa and Asia.\newline%
The implications of ANAASP and adeno{-}associated peptides are interesting.\newline%
We Have Our Chance\newline%

%


\begin{figure}[h!]%
\centering%
\includegraphics[width=120px]{./photos_from_epoch_8/samples_8_126.png}%
\caption{a man wearing a hat and a tie .}%
\end{figure}

%
\end{document}