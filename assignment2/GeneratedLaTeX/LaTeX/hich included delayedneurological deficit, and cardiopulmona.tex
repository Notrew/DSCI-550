\documentclass{article}%
\usepackage[T1]{fontenc}%
\usepackage[utf8]{inputenc}%
\usepackage{lmodern}%
\usepackage{textcomp}%
\usepackage{lastpage}%
\usepackage{graphicx}%
%
\title{hich included delayedneurological deficit, and cardiopulmona}%
\author{\textit{Mai Juan}}%
\date{04-03-2008}%
%
\begin{document}%
\normalsize%
\maketitle%
\section{A report of black hole on the links connecting increased economic growth and public services caused by the business graduates in Sri Lanka by the year 2011}%
\label{sec:AreportofblackholeonthelinksconnectingincreasedeconomicgrowthandpublicservicescausedbythebusinessgraduatesinSriLankabytheyear2011}%
A report of black hole on the links connecting increased economic growth and public services caused by the business graduates in Sri Lanka by the year 2011.\newline%
Sri Lanka’s transition to an economic model relying on a trained workforce is complicated by a vast gap in income between the employment of the men and women with particular skills and the differences in social class by which these skills differ.\newline%
In fact, the income gaps between the female and male classes are much wider and the ratio of male to female graduated graduates in Sri Lanka is higher than for the men. This disparity can result in the absence of skills, which in Sri Lanka is not transferable to those with special skills.\newline%
The report by the Joint Commission on National Development (JCD) has found the widening gap in employment between men and women in Sri Lanka. In Sri Lanka a higher number of women than male graduates is attributed to a lower number of the same graduate. This gap is particularly significant for civil servants in order to catch up with the men.\newline%
This has also led to women and their spouses being squeezed out of the classroom.\newline%
In line with JCD’s reasons for placing pressure on the Sri Lankan authorities to ensure its fiscal policies are aligned with increasing economic activity, the report shows the likely result of increased joblessness, the associated effects of the implementation of Workless Cycle in 2010, and the consequences of a lack of trained labourers.\newline%
Statistics are now available in Sri Lanka showing the emergence of a revolving door that may account for this dire situation. The report noted that if there were to be a further extension to 2010, the gap in educational attainment and training would more than double.\newline%
The higher deficit of employment among women in Sri Lanka is also pointed to by JCD, as she said that women presently placed only 20\% of employment opportunities in Sri Lanka in financial terms, a figure twice the number of men in this country.\newline%
The vast gap in educational attainment is increasingly key in efforts to plug the financial illiteracy within the lower Class of Sri Lankans, and a large and growing proportion of the workforce used to be male.\newline%
Furthering the emerging rift in educational attainment between women and the men, the JCD recommendations report urges action to end the lack of senior citizen benefits for pregnant women, and forced birth in order to encourage political stability that has facilitated the government and economic initiatives to reap the benefits of the economic expansion.\newline%
Women and girls who have graduated from university tend to report a higher educational attainment compared to female graduates, with the average age at the end of 2007 rising from 18 to 21 years. Women with tertiary education are also more likely to have a family and are less likely to depend on social support to pay for their education.\newline%
For women, a strong cultural awareness associated with achieving education in the lower and upper classes enhances their economic impact, as they are less likely to become rich personally. Studies reveal that at the beginning of the school year only 15\% of women in Sri Lanka had applied for voluntary voluntary activities such as driving taxis and reserving petrol during the electricity cut.\newline%
The same low share of women in formal education cut considerably to 12.4\% in 2002, and 16.4\% in 2005. In all these years, women had a more consistent likelihood of participating in banking services than men.\newline%
Women enjoy childcare at 30 to 32 years, with the chances of being able to work in 12\% during those years. Women also have a better economic chance of having a large household in 15\% of households.\newline%
These benefits which make the society more likely to produce and benefit society{-}wide are demonstrated by their higher earnings, estimated to have amounted to an average of \$2,000 in 2006, and by rising shares of families spending money or selling their food or cash income.\newline%
Childbirth rates in Sri Lanka have fallen from 60 in 2002 to 51 in 2005, with women taking a greater contribution to the creation of nursery sites than men. But these findings are evidence that women continue to have lower career opportunities than men, despite women having higher earnings.\newline%
M. Thales Maunelra ( is a senior researcher on education and employment at the Centre for Development and Employment Research at the University of Malaya in Thailand) is interested in highlighting the critical importance of reducing illiteracy in Sri Lanka through reduction in income inequality. He is associate professor, JCD's Member of the Royal Society of Erastums, and is board director of the Centre for Development and Employment Research at the University of Malaya. He is also chairperson of the Ministry of Education and Social Policy and policy advisor to the Ministry of Social Affairs.\newline%

%


\begin{figure}[h!]%
\centering%
\includegraphics[width=120px]{./photos_from_epoch_8/samples_8_208.png}%
\caption{a man and woman pose for a picture .}%
\end{figure}

%
\end{document}