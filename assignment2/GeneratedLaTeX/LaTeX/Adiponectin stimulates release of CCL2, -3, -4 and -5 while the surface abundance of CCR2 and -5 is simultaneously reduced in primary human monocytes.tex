\documentclass{article}%
\usepackage[T1]{fontenc}%
\usepackage[utf8]{inputenc}%
\usepackage{lmodern}%
\usepackage{textcomp}%
\usepackage{lastpage}%
\usepackage{graphicx}%
%
\title{Adiponectin stimulates release of CCL2, {-}3, {-}4 and {-}5 while the surface abundance of CCR2 and {-}5 is simultaneously reduced in primary human monocytes}%
\author{\textit{Bowen Patrick}}%
\date{11-04-2009}%
%
\begin{document}%
\normalsize%
\maketitle%
\section{by Kate Cooper\newline%
Professor of Chemistry, said there has been an increasing number of dietary food additives growing in the United States over the last few years, which are potentially toxic to humans and are further undermining the body’s immune system}%
\label{sec:byKateCooperProfessorofChemistry,saidtherehasbeenanincreasingnumberofdietaryfoodadditivesgrowingintheUnitedStatesoverthelastfewyears,whicharepotentiallytoxictohumansandarefurtherunderminingthebodysimmunesystem}%
by Kate Cooper\newline%
Professor of Chemistry, said there has been an increasing number of dietary food additives growing in the United States over the last few years, which are potentially toxic to humans and are further undermining the body’s immune system.\newline%
The proteins are part of the enzyme replacement therapy (RIPP) produced by a chemical known as CCR2 (alpha{-}C), which changes the level of the protein to CCL2.\newline%
The tissue is shaped by a dimpling of CCR2 but the fibers are thinned out, allowing CCR2 to become active and restore B cells to normal expression. However, some human nerve cells produced by CCR2 cause critical cell death and the calcium phosphate produced by CCR2 causes the surface abundance of CCR2 to slowly disappear, thereby triggering neuroses and increased amounts of CCR2.\newline%
A new study from researchers at the UCLA Earle Black Memorial Institute reveals the toxicity of CCR2 to humans’ immune system. The study provides an early glimpse of the toxicity of CCR2 protein in brains. The study also compared the evolution of gene expression in CCR2 with biological signs of CCR2 and revealed the molecule’s link to neurodegenerative diseases.\newline%
Clinical research on the natural causes of neurodegenerative diseases, including Parkinson’s disease, Alzheimer’s disease and Huntington’s disease, established the molecular profile for CCR2 in the body and tested the protein’s ability to strengthen cancerous cells and promote Alzheimer’s disease. There have been many studies on CCR2 in human brain and lipids in rodents and human humans.\newline%
Clinical research included genome{-}wide association studies of a combination of CCR2 and DNA in neurons in human neurons as well as outside a cell line, including mice, and in humans, and in their mitochondria, the engine of energy generation. Previous studies of the human genotypic proteins – the gene{-}editing neural pathways that regulate cell production – have been mixed with mixed results from different testing. A new study published in the September issue of the journal PLOS Biology enimates the safety and effectiveness of mitochondrial DNA genotyping in human human cells to serve as an early warning system for any potential neurological symptoms in mitochondrial disease.\newline%
This study and other recent studies on how CCR2 is linked to neurodegenerative diseases, including Parkinson’s disease, highlight the limitations of traditional laboratory methods and their associated human implications, said Dr. Karina Choi, UCLA lecturer in that same department.\newline%
The arms{-}length research facility is a design{-}build and pilot{-}acceleration institute where the designers and codevelopers of the new study work together to develop and test new long{-}term therapeutic hypotheses.\newline%
Dr. Choi serves as the Shepperton Professor of Metabolic and Dietetics at the UCLA Earle Black Memorial Institute in Los Angeles. Her research focuses on the relationship between genetic change and disease and explores the link between genetic changes and disease.\newline%
Other graduate students and faculty member at the UCLA Earle Black Memorial Institute include Dr. Hildebrand Mortensen, Ingber Hmeijs, Randa Gonzalez, Timothy Batten, Sandra Dene Troes, Eunice Kaufman, Phillip Tuk, Sarah Spar, Karmoa Matameth, Brendan Tinn and Robert Yervis.\newline%
\#\#\#\newline%

%


\begin{figure}[h!]%
\centering%
\includegraphics[width=120px]{./photos_from_epoch_8/samples_8_204.png}%
\caption{a man and a woman posing for a picture .}%
\end{figure}

%
\end{document}