\documentclass{article}%
\usepackage[T1]{fontenc}%
\usepackage[utf8]{inputenc}%
\usepackage{lmodern}%
\usepackage{textcomp}%
\usepackage{lastpage}%
\usepackage{graphicx}%
%
\title{en together, our findings revealan important role of proteas}%
\author{\textit{Lo Mee}}%
\date{11-04-1991}%
%
\begin{document}%
\normalsize%
\maketitle%
\section{Researchers suggest that blood transfusions can one day increase the levels of cholesterol, which contributes to heart disease}%
\label{sec:Researcherssuggestthatbloodtransfusionscanonedayincreasethelevelsofcholesterol,whichcontributestoheartdisease}%
Researchers suggest that blood transfusions can one day increase the levels of cholesterol, which contributes to heart disease. This change by slimming the body is necessary because it stops abnormal cholesterol from falling on people. The toxic effect of removing blood from the body after a stroke may cause problems with blood clotting, stroke, or other diseases as well as poor drainage of blood into the clot chambers.\newline%
The new findings, published in the journal Cardiovascular Diseases in May, by Washington University in St Louis, Missouri, suggest that chronic inflammation may play a key role in a reduced risk of heart disease.\newline%
According to the researchers, there is little or no evidence that blood transfusions increase the levels of cholesterol in the blood. However, their findings suggest that these levels may play a critical role in the development of atherosclerosis and possibly even reverse it, rather than deny it.\newline%
"In a rich animal study, researchers at Washington University in St Louis conducted a low{-}dose, high{-}dose, nutritional therapy where blood in the blood is combined with calcium and magnesium, allowing by contrast, calcium to penetrate the blood{-}producing endothelial cells in the cells where the cholesterol has no effect on blood clotting. In contrast, our targeted blood{-}carriers did not increase cholesterol levels that changed," said Martha Dunham, Ph.D., professor of preventive cardiology and member of the Center for Interventional Cardiology at the University of Washington Medical Center in Seattle.\newline%
The findings were first suggested in a study published in the journal Opinions in Clinical Cardiology.\newline%

%


\begin{figure}[h!]%
\centering%
\includegraphics[width=120px]{./photos_from_epoch_8/samples_8_235.png}%
\caption{a young boy wearing a hat and a tie .}%
\end{figure}

%
\end{document}