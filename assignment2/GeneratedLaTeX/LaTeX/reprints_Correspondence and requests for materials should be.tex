\documentclass{article}%
\usepackage[T1]{fontenc}%
\usepackage[utf8]{inputenc}%
\usepackage{lmodern}%
\usepackage{textcomp}%
\usepackage{lastpage}%
\usepackage{graphicx}%
%
\title{reprints\_Correspondence and requests for materials should be}%
\author{\textit{Wang Zhu}}%
\date{04-27-1990}%
%
\begin{document}%
\normalsize%
\maketitle%
\section{Many printers now provide a variety of stories to someone interested in their printing process}%
\label{sec:Manyprintersnowprovideavarietyofstoriestosomeoneinterestedintheirprintingprocess}%
Many printers now provide a variety of stories to someone interested in their printing process.\newline%
Just ask Daniel Christian, a local printer and the owner of Lincoln Bangon, One Bill Book, of Toonide, South Africa. The man also works for HFC Printing.\newline%
“He looks to me as if he is representing something by printing something just to get some advertisement in return,” he says.\newline%
{[}He{]} has been doing printing in the past but has not needed as much money or as much printing,” Christius acknowledges.\newline%
“He has not forgotten about this right now… He has been telling people in some unnamed South African village and asking them, ‘Where will you choose to print?’ he says.\newline%
“He has not had to pay for the document!” Christius said.\newline%
Many of the stories show how no matter what anyone finds, it will be produced by someone paying with pennies and coins.\newline%
“Most people run away from anything – the toilet, the roof of your car, the telephone lines, the sick or elderly when they need to have a replacement,” Christius explains.\newline%
Many of those stories have produced a review or first edition of several newspaper stories, usually published by someone advertising them to print.\newline%
There are also newspapers, web portals and publishers whose advertising does not usually translate to anywhere else.\newline%
“It’s just that if people can’t believe what they are seeing at the moment in terms of the individual stories and their own aesthetics, that is the truth,” Christius says.\newline%
He recalls the story of a young black man who told him about not having his full name printed because he couldn’t afford the product.\newline%
“It was usually people saying there was a branch around the town selling some take out until it was done,” Christius says.\newline%
Another event in an interesting way illustrates how printing is spreading to more places and especially in rural communities.\newline%
A book by a prominent Burundian man is being circulated in an attempt to raise money for his family’s clothes and clothing company.\newline%
The new paper works as an advertisement of print and printing, displaying a large picture of the words ‘Airtray’ which means black.\newline%
A second of the titles announces that it is being printed with a picture of the word ‘Airtray’ which is printing under a commercial name.\newline%
“We think that is an authentic ad which is the most authentic example of printing,” Christius says.\newline%
The colour of the print will not be too bright.\newline%
If you need information on how to request materials for printing in the area, contact me on 6129 30 74.\newline%

%


\begin{figure}[h!]%
\centering%
\includegraphics[width=120px]{./photos_from_epoch_8/samples_8_181.png}%
\caption{a man and woman pose for a picture .}%
\end{figure}

%
\end{document}