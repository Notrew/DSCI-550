\documentclass{article}%
\usepackage[T1]{fontenc}%
\usepackage[utf8]{inputenc}%
\usepackage{lmodern}%
\usepackage{textcomp}%
\usepackage{lastpage}%
\usepackage{graphicx}%
%
\title{A Neoplastic Gene Fusion Mimics Trans{-}Splicing of RNAs in Normal Human Cells}%
\author{\textit{Mahmood Mohammad}}%
\date{11-27-1999}%
%
\begin{document}%
\normalsize%
\maketitle%
\section{Neoplastic RNAs are transpolar cells that replicate themselves independently}%
\label{sec:NeoplasticRNAsaretranspolarcellsthatreplicatethemselvesindependently}%
Neoplastic RNAs are transpolar cells that replicate themselves independently. This helps them achieve the function normally of typical human cells. The technology addresses several core medical problems that include spinal deformities and difficulties associated with normal human growth and development. The first gene study has been carried out in neurons of somatic cell{-}dependent nurses. The study has been presented at the American scientific meeting of the American Institute of Neurological Research (AIDE) in California in November 1999.\newline%
More recently, a similar clinical trial has been carried out in cytochromatocytes, a group of cells of the brain that consists of 13 types of cells, with loss of function, the function of the human trachea, as well as neuropathic pain. This is a typical treatment for neuropathic pain and disorder. From a clinical perspective, the researchers consider the fact that according to the AIDE Guidelines the Neuropathic Pain Disorder course is a relatively new manifestation of neuropathic pain since the monophysoid syndrome was experienced in Europe in the 1970s. In all, more than 110 people in the study, previously untreated neuropathic pain and nerve pain experience a symptom of neurological related pain while the epileptic pain themselves. The study is based on initial findings about 985 neuropathies, 22 patients and 22 patients over a 12{-}week period, and its results, which included pro{-}inflammatory sites, may offer insights into the limits of the neuropathic pain.\newline%
The study is part of efforts to enhance neuropathic pain in humans. In a review of the human meta{-}analysis of brain scans of rats, the researchers concluded that the presence of neuropathic pain is “inexpensively associated with causes of neuropathic pain and associated with physical inflammation of the brain.” They were also concerned about the extent to which the brain area may suffer from neuropathic pain. Therefore, the authors considered the possibility that these types of induced neurotoxicity – known as “electrophysical induced neurotoxicity” – can be employed in humans to curb neuropathic pain in humans. “Patients do not appear to have neuropathic pain and treat it with a simple pain remedy,” the authors wrote. The study was published in the current issue of Neurosurgery in Asia where it is entitled: “Neoplastic Biological Disturbances: Methods for Neuropathic Pain.”\newline%
Researchers at the University of Minnesota Duluth School of Medicine and University of California, San Francisco, released their own article on the trial’s results. According to Kevin Pachma, a psychiatry researcher at the University of Minnesota in the United States, “It appears to be contrary to the overall theory of neuropathic pain.” He acknowledges that several new gene studies that were reported about the trial, have failed to provide any benefit.\newline%
Proceedings were co{-}authored by Andrew Lacy, Editor in Chief of Neurosurgery in the journal Neurosurgery (American Psychiatric Association) and Megan Perlis, Biologist in the Department of St. Luke’s and St. Luke’s University School of Medicine in Minneapolis, Minnesota. The study was supported by the National Institute of Neurological Disorders and Stroke (NIH, United States Department of Veterans Affairs), American Institutes for Research, and Veterans Administration.\newline%
M\newline%

%


\begin{figure}[h!]%
\centering%
\includegraphics[width=120px]{./photos_from_epoch_8/samples_8_200.png}%
\caption{a man in a suit and tie holding a cell phone .}%
\end{figure}

%
\end{document}