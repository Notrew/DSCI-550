\documentclass{article}%
\usepackage[T1]{fontenc}%
\usepackage[utf8]{inputenc}%
\usepackage{lmodern}%
\usepackage{textcomp}%
\usepackage{lastpage}%
\usepackage{graphicx}%
%
\title{) fragmentation\_ They disappeared upon co{-}treatment with wor}%
\author{\textit{Liu Lien}}%
\date{03-20-2003}%
%
\begin{document}%
\normalsize%
\maketitle%
\section{For those who prefer not to alienate the 3}%
\label{sec:Forthosewhoprefernottoalienatethe3}%
For those who prefer not to alienate the 3.0 communication system operating in the hypervisor, well in the near future, taking steps forward in development of unified communications and a full unification of the Internet system will be instrumental. They will soon figure out what such a system would look like, and they will shift from support for point{-}of{-}sale systems into support for database management. That much is certain.\newline%
On the surface, jolting change is only natural. Consumers will see a need to integrate their mobile handsets and other communications boxes in the same way a converged network wires a network together. This will ensure seamless interoperability between all these systems, from the banking kiosks to the application servers. It will make it easier for web connectivity, communications, and other shared services.\newline%
But there is a deeper, more important problem. There is an interconnectivity problem. Browsers offer only two or three identifiers, which effectively render all communication being copied into one another or crossing over other traffic.\newline%
Control managers are hard to create and fix, and they are yet another burden of convergence that would dominate communication in the wake of interoperability solutions. For that, there is still an essential level of interoperability that must be moved forward.\newline%
Objects {-}{-} the video feed, the phone calls, as well as services, such as calling numbers and Internet access {-}{-} remain unshackled, made less dependable, vulnerable to wrongdoings, or absent from account activity. This is the key to solving interoperability issues and building a more unified communication strategy. The only way to build that is by making a simple component of each.\newline%
Local carrier proxies help to facilitate interoperability of network traffic in the underlying infrastructure, with carrier directories providing a list of geographic locations and service area. It means that carriers can move information into the closest data center and be insulated from internal interference. It means that the interface between communication systems and their computers is symmetrical and that data, as well as voice transmissions, can be used almost identically.\newline%
And both these are a genuine requirement to interoperate fully. Network network infrastructure should primarily be managed by carriers. But who knows what networks will look like once these two vendors, the Gola Layer and Wireless Garage, bring together together the systems of the governments, the telephone companies, and, eventually, the major players.\newline%
To maintain those interests, not just on technical terms but also politically, each carrier, the handset manufacturer, the cellular operator, and the phone company need to adopt or should adopt unified communications at every level of every organization.\newline%
It is important that the various carriers, though, stick to the stated objective of the two vendors, namely to support the integration of their systems. This not only means sharing and remembering databases or network information, but not just terminating the browser server and setting up services based on whether or not they can handle being able to see the data.\newline%
The key to the interoperability process will be the level of relationship between the carriers, the network network operator, and the wireless operator. Everything on the network or on the network is connected, but the cell service provider and the phone company still have to enforce only partial compliance with only partial rule orders. Nobody will build to enforce against those in complete compliance with the same laws.\newline%
The solution may involve having multiple networks connected to one another to form a single interoperable network. Three interoperable networks will provide access to each other and the carrier can benefit from fostering an environment of seamless mobile and Internet connectivity.\newline%
There are an infinite number of development techniques and pathways, not just for the reality of the relative strengths of two vendors, but also the size and efficiency of the companies that are in the business. The problem is, to arrive at an interoperable network it has to fall over each other.\newline%
From a technical point of view, the one advantage for choosing vendors is that they have seen very large cross{-}platform adoption over the past 10 years. The key is to produce, and distribute, results that will come out in the open market. The ability to experiment could be the key to improving the quality of interaction between information and the carrier.\newline%

%


\begin{figure}[h!]%
\centering%
\includegraphics[width=120px]{./photos_from_epoch_8/samples_8_453.png}%
\caption{a woman in a red shirt and black tie}%
\end{figure}

%
\end{document}