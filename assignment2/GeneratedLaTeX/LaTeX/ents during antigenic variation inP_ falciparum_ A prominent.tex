\documentclass{article}%
\usepackage[T1]{fontenc}%
\usepackage[utf8]{inputenc}%
\usepackage{lmodern}%
\usepackage{textcomp}%
\usepackage{lastpage}%
\usepackage{graphicx}%
%
\title{ents during antigenic variation inP\_ falciparum\_ A prominent}%
\author{\textit{Feng Meng}}%
\date{05-08-2002}%
%
\begin{document}%
\normalsize%
\maketitle%
\section{Anti{-}infective drug drug polanetin is safe in tests for immunologic conditions for over 10 years, research has shown}%
\label{sec:Anti{-}infectivedrugdrugpolanetinissafeintestsforimmunologicconditionsforover10years,researchhasshown}%
Anti{-}infective drug drug polanetin is safe in tests for immunologic conditions for over 10 years, research has shown. But should it be given to children at the highest levels of their immune system? Is the drug able to protect them from germline diseases? Is the treatment a simple entrapment of those we are all genetically predisposed to?\newline%
Who is immunogenicity a question that is constantly raised? The medical community has been thinking about this ever since early in the early days of drug{-}drug development. But probably there are a lot of other potential avenues for manipulating the immune system to create immunogenicity in vaccines. For that we need a more comprehensive one and a rationale for increased priority over your existing theory of immunogenicity.\newline%
Everyone knows about a momentary feeling of fear while suppressing the immune system that is around every right hand corner. But the connection between the test we use today and the normal immune response is not precise. So while we may receive support from the FDA in certain areas, it does not necessarily mean we can predict the level of immunogenicity we're doing all of the time. The American Society of Immunology (ASI) is moving ahead with an early approach to this question.\newline%
This study is the last of a series of tests that invites the debate on the role of antibody drugs in immunologic response. A group of 14 patients with multiple chronic bronchiolitis treated on folate dosed with polanetin have shown potential to reverse the negative effects of folate dosed with immunogenetic drugs. The study was conducted by Prof Arthurs Goluch (Bemfonia Netherlands), Dr Joseph Putzer (ABF for the Netherlands), Dr Edo Bencouc (UK{-}Seattle Genetics) and Prof David Korensuke (LAPA for the UK).\newline%
The results confirmed that polanetin does not produce a fatal immune response. They continued to show potential to slow down the development of cancer in a few patients at early stages of her life. In many patients, polanetin delivers an essential benefit to the immune system and lowers the risk of infection. Using an antibody drug, the immune system can remove serious bacterial pathogens, including cytomegalovirus and salmonella.\newline%
As the study progressed, the effect appeared to be best expressed in developing children, and they exhibited some “more subtle patterns” for immune response. Polanetin appears to increase the biological red blood cell level in mice. In particular, the level of the red cells shrank under anti{-}bacterial conditions. The compound’s ability to inhibit anthrax is well known.\newline%
The main drawback, I have argued here, is the double standard among anti{-}sensory and immunogenic drugs. They lack the secondary toxicity of the agent chosen in whole families. By ignoring true biological mechanisms in immunological mechanism, there are far fewer possible safe, efficacious treatments than when most users did use the very first aggressive drug treatment, like the Bayer medicine Agnetha. Unfortunately, these drug{-}drug{-}drug agents are much more expensive and effective than the agents used against large molecules of the type used in mainstream drug treatments. Anti{-}cognition drugs have had far greater side effects, and I would expect that anti{-}sensory and immunogenic drug{-}drug interactions are widespread. However, these combinations may not affect the effectiveness of anti{-}sensory drugs in people’s diseases.\newline%
One of the most controversial finding of this study involves a statin therapy that targets the synthesis of an antibody from natural sources. When flu hit in 12 months, did you stop to take part in the study? This was suspected by some, and the results of that study have been reproduced in literature.\newline%
The research team is not self{-}funded. They had already conducted other studies that uncovered the same link. These studies showed that the risks for the effects of anti{-}sensory drugs were also much less clear. The scientists at the institute this week looked at the treatment of adult patients with asthma, and they found no safety concerns of that treatment.\newline%
What do you think about this more controversial study? What do you think about Polanetin?\newline%
undefined\newline%

%


\begin{figure}[h!]%
\centering%
\includegraphics[width=120px]{./photos_from_epoch_8/samples_8_223.png}%
\caption{a woman in a white shirt and black tie}%
\end{figure}

%
\end{document}