\documentclass{article}%
\usepackage[T1]{fontenc}%
\usepackage[utf8]{inputenc}%
\usepackage{lmodern}%
\usepackage{textcomp}%
\usepackage{lastpage}%
\usepackage{graphicx}%
%
\title{stasis is a primary cause of cancer death\_ Antrodia cinnamom}%
\author{\textit{Fu Fang}}%
\date{03-20-1997}%
%
\begin{document}%
\normalsize%
\maketitle%
\section{Malawi's first case of cancer appeared at the Consumer Health Technology Assessment (CCHA) national screening, said Antrodia National Cancer Foundation leader Elizabeth Paliga}%
\label{sec:MalawisfirstcaseofcancerappearedattheConsumerHealthTechnologyAssessment(CCHA)nationalscreening,saidAntrodiaNationalCancerFoundationleaderElizabethPaliga}%
Malawi's first case of cancer appeared at the Consumer Health Technology Assessment (CCHA) national screening, said Antrodia National Cancer Foundation leader Elizabeth Paliga.\newline%
"She (Mandakalio) was found to have cancer that has spread to her brain. People who are very healthy and have immunity are at risk from this disease," Paliga said.\newline%
"If the scan did not show the cancer, then people should not worry about her symptoms."\newline%
Paliga said the well{-}known procedure is meant to contain the immune system from the face of cancer, with little impact to it.\newline%
Deaths to or cancer{-}associated cancers\newline%
What took place at the clinical assessment was the first case of suspected cancer in Malawi, with another case found near the beginning of the year.\newline%
The case involved Mutota, aged 50, who was found to have caused illness from early age. She had relapsed from multiple myeloma at about 40 weeks, discovered by her doctors a few months earlier in a bone marrow transplant and subsequently had a hip bone and a knee operation.\newline%
With the transplant already taking place, the consultant{-}led bone marrow transplant was not possible.\newline%
She was initially treated for a buildup of red blood cells in her lungs, however, after a scan in November 1997 the disease appeared to be returning.\newline%
Conditions were worsening before she was admitted to hospital, according to the outpatient management group, which conducted the review.\newline%
"She developed a severe bronchitis that increased the risk of the infection and ultimately became cancerous," said Carlos George, group general manager.\newline%
Medics concluded that after an individual had been diagnosed with mesothelioma in 1995, her illness had been moving, which had been the cause of her prolonged survival.\newline%
"She was cured, but obviously her treatment didn't do her a great favour," George said.\newline%
"That led to another trial to determine whether she had become resistant to chemotherapy."\newline%
International study\newline%
Paliga said that such advanced disease is being tracked by Angolan experts at the World Health Organisation (WHO) because of its low levels of data.\newline%
"As with any disease, everyone needs a dose of antibiotics," she said.\newline%
"There are many unknowns about how to manage the high doses of antibiotics without reducing their resistance. The best approach will be more effective to treat the tumour, and to improve the quality of life of infected people and sustain the immunity against a recurrence."\newline%
Paliga stressed that these are not the only reasons why it is important to follow up the initial diagnosis.\newline%
"The survival rate depends on all the factors listed in the annual results," she said.\newline%
"But an individual needs to be managed in a multi{-}solutionised way based on cost and need. We know from the National Regional Lead UK Cancer Centre (NECHC) that there has been a 30 per cent increase in the survival rate after the test performed."\newline%
She said a programme such as Lung cancer treatment was aimed at progressively reducing the risk of the disease, for example by moving more patients who are living in shanty communities from the critical clinic location of the cancer hospital.\newline%
The most popular treatment involves the removal of the small intestine within a patient's tumour, which benefits the patient considerably.\newline%
The importance of a key test can be gauged by comparing the results of the team at the national laboratory in Malawi with another Angolan medical community.\newline%
Epithelial cells (the stages of growth of adult cancer) are also very useful in cancer treatment, with a small amount containing a protein called anthocyanin (a type of cell that is particularly beneficial in alizative chemotherapy).\newline%
Malawi is a country with a wealth of documented health problems, ranging from the several{-}million Angolan refugees living in temporary camps to the treatment of HIV and cancer.\newline%
The highest incidence of cancer is in the event of malaria and diarrhoea, and usually people die within one to three years.\newline%

%


\begin{figure}[h!]%
\centering%
\includegraphics[width=120px]{./photos_from_epoch_8/samples_8_456.png}%
\caption{a man in a suit and tie standing in a room .}%
\end{figure}

%
\end{document}