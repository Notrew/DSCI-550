\documentclass{article}%
\usepackage[T1]{fontenc}%
\usepackage[utf8]{inputenc}%
\usepackage{lmodern}%
\usepackage{textcomp}%
\usepackage{lastpage}%
\usepackage{graphicx}%
%
\title{ine the content in such documents, for the purposes of acade}%
\author{\textit{Kao Bao}}%
\date{08-15-1997}%
%
\begin{document}%
\normalsize%
\maketitle%
\section{POWERTIS SAYS he will warn members of the community not to disclose their views at events}%
\label{sec:POWERTISSAYShewillwarnmembersofthecommunitynottodisclosetheirviewsatevents}%
POWERTIS SAYS he will warn members of the community not to disclose their views at events.\newline%
For Patrick Calin, the recently re{-}elected MP for Troop Byron, the problem with these documents is not that they are a threat to our democracy. For the majority of our citizens we have significant political parties in our communities.\newline%
We know that we are all susceptible to opinions. Yet in our market of political parties and its interests we have been forced to accept ideas that are disputed. And this fear often involves various forms of view, which use many different definitions, implications and interpretations.\newline%
Whether strongly held or unpopular with our communities, persons who decide on how to form political parties freely discuss their views. They often act from a position of credibility, and they usually assume they know who the other party is.\newline%
Sometimes they are also acting from a position of comfort, when they have also assumed they know one another. Or there may be members of the public who decide to abandon their political opinions, but who who also walk away instead of attempting to exercise meaningful influence over the next step and decide a path forward.\newline%
Sometimes evidence is presented to demonstrate a party's reputation of overreaching. But are they simply valid attempts to control or inhibit debates or send a message to the public that they hold a different view than others?\newline%
The Greeks suffered heavily from this very phenomenon. Their political parties were monopolised by political parties in the form of parties with the strongest divisions on the basis of their views. Their plans for political leaders focused on strengthening their interests and not on the future of society.\newline%
That is why the Cyprus Amendment to the Constitution was in such instances essential in forcing a dialogue, and it is also why some, such as the British prime minister John Major, generally commented about the letters of agreement left in Parliament's summary judgment, which nonetheless made a major contribution towards the preservation of democracy in our country.\newline%
I have not yet read a politician's written statement making these points. Perhaps I am losing faith in such figures. I know every member of my constituents are opposed to their representation. Some of them are for some reason fine with the idea that they support a group only that holds a specific social position such as one that would define a person as a private person. But they would rather do what they do.\newline%
The future of our nation is very difficult. But with this generation having grown to be citizens and members of parliament, it is frankly unjust to place such political beliefs in the hands of so many.\newline%
Patrol.co.za\newline%

%


\begin{figure}[h!]%
\centering%
\includegraphics[width=120px]{./photos_from_epoch_8/samples_8_433.png}%
\caption{a man in a suit and tie is smiling}%
\end{figure}

%
\end{document}