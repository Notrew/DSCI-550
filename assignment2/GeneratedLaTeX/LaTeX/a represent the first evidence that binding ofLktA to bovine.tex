\documentclass{article}%
\usepackage[T1]{fontenc}%
\usepackage[utf8]{inputenc}%
\usepackage{lmodern}%
\usepackage{textcomp}%
\usepackage{lastpage}%
\usepackage{graphicx}%
%
\title{a represent the first evidence that binding ofLktA to bovine}%
\author{\textit{Li Yong}}%
\date{07-16-2001}%
%
\begin{document}%
\normalsize%
\maketitle%
\section{When Dani Kawashima said she found their video ‘Matilda’ on her desktop yesterday, it immediately attracted a lot of attention and conversation}%
\label{sec:WhenDaniKawashimasaidshefoundtheirvideoMatildaonherdesktopyesterday,itimmediatelyattractedalotofattentionandconversation}%
When Dani Kawashima said she found their video ‘Matilda’ on her desktop yesterday, it immediately attracted a lot of attention and conversation. No doubt it was ‘comical’. Her belief was that all Bulgarians and Jews share a common sense of one common language.\newline%
Not only was it beautifully detailed and well{-}written, but the reference to the word ‘teze’ inscribed on ‘80’s film posters of local World War 1 trucks is also quite a concept. What defines the only African country with this advantage is its history of concentration camps. In all of its forms, this little flick also has a strong resemblance to the infamous cartoon Bokassa.\newline%
Even though we may have changed our perspective a little over the years (see Julia Ormond’s Irish comedy, Livin’ Wild), it all was accepted by European audiences as the subject of the film. I believe it makes sense to own ‘Matilda’ (Picador, July 12, 2000) so it’s unlikely that it will be forgotten for some time. The full answer will be the film’s source material.\newline%
It’s by contrast another film that I believe will become familiar to many once it is given an uncensored theatrical release. I’ve been playing a lot of this film in the past few years, which explains the authenticity of the film. It was sent by a small documentary company that was behind the documentary produced by Vidya Sarmadou for Kyunki Mouni Variety which is in the subject of this documentary. The film features women from the Congo speaking Kiswahili and Mandarin in the studio and much of the footage is of our arrival back home.\newline%
There is a real element of anticipation, especially in the company of all of us. Because this film is about the life of Mega Dawaitop (a retired police officer who, as we are all aware, is not very interested in safety or the privacy of others as demonstrated by her husband), we felt we had a chance to share the best film we had seen. The look on Mega’s face after seeing the documentary, and the manner in which she managed to keep the secret is just amazing.\newline%
Mega Dawaitop is one of those people who is as good as her word and as proud of her work as everyone else in her native country. And this is about Sherai Phrubikomi, a farmer farmer here. A strong supporter of the Polizei and other African liberation movements, Dawaitop is more passionate about independence, ensuring that the work of the Polizei in Democratic Republic of Congo is honoured by the masses.\newline%
What is needed to make this documentary possible is a theatrical release and the support of industry executives, helping make it possible for it to be shown on a major European level. This will be the template for other documentaries, short films, something an untrained viewer may find difficult to ignore. I’d love to see this film released to the media. I’m wondering how it is this lack of publicity leads to the arrest of the film’s producer. I’m hoping this information will help bring this film to the international media soon.\newline%

%


\begin{figure}[h!]%
\centering%
\includegraphics[width=120px]{./photos_from_epoch_8/samples_8_209.png}%
\caption{a man in a suit and tie is smiling .}%
\end{figure}

%
\end{document}