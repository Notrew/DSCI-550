\documentclass{article}%
\usepackage[T1]{fontenc}%
\usepackage[utf8]{inputenc}%
\usepackage{lmodern}%
\usepackage{textcomp}%
\usepackage{lastpage}%
\usepackage{graphicx}%
%
\title{bridge, UK8VIB Switch Laboratory, Department of Cellular and}%
\author{\textit{Sun Shen}}%
\date{07-12-2006}%
%
\begin{document}%
\normalsize%
\maketitle%
\section{Cisco vu And we are back again\newline%
The year is 1996 and the network of the UK are on a new road}%
\label{sec:CiscovuAndwearebackagainTheyearis1996andthenetworkoftheUKareonanewroad}%
Cisco vu And we are back again\newline%
The year is 1996 and the network of the UK are on a new road. At launch, 4,720 villages were left waiting for the Ireland{-}Connected Geo that would give them everything they need, connecting their home and business with a new way of living.\newline%
The new piece was advanced by Connex, the network of self{-}powered switch labs co{-}founded by experienced technologists Martin Lowrance and Marina Perlo. Their efforts have since multiplied in popularity. And, over the past two months, they have enabled some 8,720 villages across all home, business and indoor markets.\newline%
In order to achieve a successful launch, the Lab has been in competition with Sonatharth Communications Ltd and Network England (NEC). Sonatharth, of Hohenzigelt on Germany’s east coast, is a partner in the network development and sales operations.\newline%
It is fairly easy to set up a network and network manager and takes a little more money on the equipment. However, the core experience is that the network should bring many demands on the company including large capacity volumes and large internet speeds.\newline%
As the results from the UK show, this has not only been positively as a result of new innovation, the local community has embraced the network, causing systems such as DFM Hub to completely revolutionise the way they compete with home routers, the portables, and dishwashers and pumps.\newline%
But more important than the success of the UK has been the exponential growth in the number of PCs available. While we've seen massive growth in terms of netbooks and the Windows 8 family, the UK has been hit the hardest when it comes to netbooks and DSL equipment. In fact, the average UK home home is more likely to have two PCs, each shipped with mobile handsets, than a total of seven PCs per household.\newline%
At the same time, on average the UK’s business communities are seeing a long{-}term bump in digital home networking being used over a wider range of desktops and smartphones, compared to a decade ago. The area of innovation that they are creating provides what I’m suggesting is a recipe for success.\newline%
The UK does have a way to go, but there is hope that we could reinvent the way that mobile users and organisations work together in a broadband connected world. If that is possible, then it is clear that the opportunities are here to be had.\newline%

%


\begin{figure}[h!]%
\centering%
\includegraphics[width=120px]{./photos_from_epoch_8/samples_8_19.png}%
\caption{a close up of a person holding a baseball bat}%
\end{figure}

%
\end{document}