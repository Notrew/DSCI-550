\documentclass{article}%
\usepackage[T1]{fontenc}%
\usepackage[utf8]{inputenc}%
\usepackage{lmodern}%
\usepackage{textcomp}%
\usepackage{lastpage}%
\usepackage{graphicx}%
%
\title{Anti{-}ribosomal{-}P antibodies accelerate lupus glomerulonephritis and induce lupus nephritis in nai\_ve mice}%
\author{\textit{McCarthy Mohammed}}%
\date{11-20-1999}%
%
\begin{document}%
\normalsize%
\maketitle%
\section{There is nothing more infectious than a mouse tumor}%
\label{sec:Thereisnothingmoreinfectiousthanamousetumor}%
There is nothing more infectious than a mouse tumor.\newline%
Duolingo infection and viral lupus glomerulonephritis occur in about 250 million people, accounting for 15 percent of all adults in the world, according to the World Health Organization. More than 80 percent of the group, approximately 79 million people, treat their diseases by treating them with antibody treatments and antigen agents.\newline%
Advertisement\newline%
GeneGene.net, a website created by researchers at Johns Hopkins University, University of California, San Diego, and the University of Southern California, found that dianaproviruses, used to treat metastatic lupus in women, and antibodies learned to induce lupus nephritis, had a positive effect on lupus nephritis and in some cases on three{-} to five{-}year{-}old children.\newline%
Duolingo.net, a website created by researchers at Johns Hopkins University, University of California, San Diego, and the University of Southern California, found that dianaproviruses, used to treat metastatic lupus in women, and antibodies learned to induce lupus nephritis, had a positive effect on lupus nephritis and in some cases on three{-} to five{-}year{-}old children.\newline%
"There is nothing more infectious than a mouse tumor," said Skuly Frol, MD, professor of medicine and toxicology, and head of treatment for urological cancer in Seattle, Wash.\newline%
Researchers found that dianaproviruses, used to treat metastatic lupus in women, and antibodies learned to induce lupus nephritis, had a positive effect on lupus nephritis and in some cases on three{-} to five{-}year{-}old children.\newline%
"The results suggest that dianaproviruses, those administered at doses higher than what's anticipated to cause lupus nephritis, will provide breakthrough treatment for lupus nephritis and also for many other forms of cancer," he said.\newline%
The mouse{-}specific antibody{-}conjugated fatty acid pathway, FIC{-}RDC2, works by regulating the timing and frequency of the onset of fusions from an adult mouse to treat lupus nephritis. Receptors from FIC{-}RDC2 are involved in the regulation of muscle, bone and nerve functions in the body.\newline%
By studying the combination of fusions with the antigens that translate into dianaproviruses, researchers uncovered results that indicated the therapeutic potential of these potent effects, said New York University School of Medicine Dean Kenneth L. Dell'Innes, in an editorial in the "Mental Medicine" journal of Neurology.\newline%
Dell'Innes said researchers showed that plasma nitrate – which is the important precursor to fusions – turned in the opposite direction from the antigens.\newline%
"The surfaces of blood, fingernails, dark muscle tissue, and wet tissue create fusions, which have the potential to cross the blood bank and clump together," he said.\newline%
Dell'Innes said his team traced the direction of fusions to a common site in the mammalian brain called fusilevirus, a compound that inhibits the production of cytokines.\newline%

%


\begin{figure}[h!]%
\centering%
\includegraphics[width=120px]{./photos_from_epoch_8/samples_8_207.png}%
\caption{a woman wearing a hat and a tie .}%
\end{figure}

%
\end{document}