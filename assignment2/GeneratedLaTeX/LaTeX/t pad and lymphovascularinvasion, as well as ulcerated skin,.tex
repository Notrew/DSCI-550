\documentclass{article}%
\usepackage[T1]{fontenc}%
\usepackage[utf8]{inputenc}%
\usepackage{lmodern}%
\usepackage{textcomp}%
\usepackage{lastpage}%
\usepackage{graphicx}%
%
\title{t pad and lymphovascularinvasion, as well as ulcerated skin,}%
\author{\textit{Niu Lok}}%
\date{08-05-1991}%
%
\begin{document}%
\normalsize%
\maketitle%
\section{Theatre International, Zimbabwe has appointed Dr Sarah Ramsden as general practitioner and Dr Tambo Mahoma as head of business development and general manager}%
\label{sec:TheatreInternational,ZimbabwehasappointedDrSarahRamsdenasgeneralpractitionerandDrTamboMahomaasheadofbusinessdevelopmentandgeneralmanager}%
Theatre International, Zimbabwe has appointed Dr Sarah Ramsden as general practitioner and Dr Tambo Mahoma as head of business development and general manager.\newline%
At the same time, the museum has upgraded its media education programmes. Dr Ramsden, who is also a doctor in the US, had worked at the same doctor’s clinic in Kingston before making her decision to enrol her in theatre from an early age, she told i, i,  .\newline%
Dr Ramsden said the spirit and culture of theatre is at a crossroads. There is now a greater need for the arts, especially on the dance floor, that is what the theatre institute is looking to implement.\newline%
“It is so long ago. It is a start of something, and everyone can learn to play it,” she said.\newline%
Dr Ramsden, who is also a medical doctor and also has a PhD in neurosurgery and molecular lumbar spine, said theatre has long been an exciting way of teaching people theatre.\newline%
“It is important to know how to play and how to play. We have different ways of teaching theatre, but when it is really done, it is very pleasing,” she said.\newline%
Dr Ramsden is confident theatre will see increasing occupancy in the coming months.\newline%
“We have grown in love with the theatre with our patrons. It is a wonderful place to work. If we can identify the theatre’s identity, then we know our patrons enjoy it,” she said.\newline%
Dr Ramsden, who recently visited the Leland Kabusi Primary School to assess needs for building access, said both theatres are in “essential” phases.\newline%
“Analysed and preparedness as well as developed learning conditions can all be a big success in identifying and improving theatre in schools,” she said.\newline%
She said the impact is being felt in other settings too.\newline%
“Churches, some people may live in neighbourhoods or places they can’t easily access, but performances and performances into their homes are very rewarding,” she said.\newline%

%


\begin{figure}[h!]%
\centering%
\includegraphics[width=120px]{./photos_from_epoch_8/samples_8_129.png}%
\caption{a man in a suit and tie is smiling .}%
\end{figure}

%
\end{document}