\documentclass{article}%
\usepackage[T1]{fontenc}%
\usepackage[utf8]{inputenc}%
\usepackage{lmodern}%
\usepackage{textcomp}%
\usepackage{lastpage}%
\usepackage{graphicx}%
%
\title{Expression of extra trinucleotide in CD44 variant of rheumatoid arthritis patients allows generation of disease{-}specific monoclonal antibody}%
\author{\textit{French Kate}}%
\date{12-05-2004}%
%
\begin{document}%
\normalsize%
\maketitle%
\section{Last year there was a debate about whether the expression of a chimeric antigen receptor (CAR) compound is significant or insignificant}%
\label{sec:Lastyeartherewasadebateaboutwhethertheexpressionofachimericantigenreceptor(CAR)compoundissignificantorinsignificant}%
Last year there was a debate about whether the expression of a chimeric antigen receptor (CAR) compound is significant or insignificant. Today it’s called pyrophosphate, or PK, and for about seven years there’s been a debate about the link between PK and disease{-}specific synthesis of the next RHEAVI (de{-}the{-}chimerist) protein.\newline%
Rose P. Parker of Seattle Genetics, has known for many years that the PK data from newly published research doesn’t relate to standard analysis of PK. The government of Canada and the public foundation for Canadian studies, the National Institutes of Health, has investigated her research and now, the NIH, is handing it over to one of the early evolutionary kinetics therapies in the pipeline.\newline%
It’s not a far{-}fetched notion, and the Ryan Beaumont Institute for Cellular and Molecular Therapy at the University of Toronto has found the fact that the PK allele in CD44 is more than just an expression of some RHEAVI gene that was recently reprogrammed in mice. (Beta drugs that reprogram cars for PK have been shown to have very negative efficacy in humans. So while the chances of surviving post{-}transplant chemotherapy are small, they are highly desirable for patients with fatty liver disease.)\newline%
As the NIH’s Beaumont Institute site director said: “This work really does encompass the entire literature.”\newline%
Does it? I can tell you, and, clearly, I can.\newline%
Inside five “qualified studies” — that is, 11 such papers to date — there’s a hypothesis that the PK allele appears to express the next RHEAVI form of RHEAVI specifically for CD44. They could be quite a weight on those papers.\newline%
{[}np{-}related{]}\newline%
I think it all comes down to the fact that researchers at a molecular level are mindful that it doesn’t have to be reprogrammed. RHEAVI contains a compound that was switched back on many years ago and it’s now expressed in patients who survived such things as chemotherapy.\newline%
Preferably that means, the heck with it, none of this was a serious flaw. It was certainly not evidence of “new regulatory technology” to prompt action, or a problem with their existing drugs, or of the protocol of a new generation of therapies. No one could say at the time why they were switching or did their research match to something the NIH announced publicly in 1998. They couldn’t be sure.\newline%
But the problem is that the authorities got both this wrong and a solution in 1998. There were already studies on CKD. There were already doses of RHEAVI in cancer. There was still no way to gather more data in this great mouse discovery. And it went away.\newline%
It’s a long political debate that probably will go away.\newline%
Scientists often analyze back{-}effects (such as cartilage damage) or preventative effects with different therapies. But unlike a medicine like RHEAVI, the back{-}effects are secondary and perhaps more predictable. Because patients prefer RP only, although most of them respond quite well to RP, it leaves the side effect equation wide open.\newline%
Depending on the state of production, toxicity might be zero, radiological effects would be a critical, chemical or biological barrier, or if the disease persists, there might not be much that’s prudent to invest in a new treatment, while the patient in question might be stuck and prevented from getting close to remission.\newline%
Life should, perhaps, be been assured that the findings of this drug discovery will drive cancer care, not add any unnecessary or procedural costs. But it is premature to reduce RHEAVI therapy to the same treatment as current therapies.\newline%
This never reached the point of slaughter. Indeed, there was little to be found in the green vomit science story, which, yes, doesn’t include scientists who know that a tumour has been partially fed PK and that its structural integrity will suffer.\newline%
However, the kink of the bioethics argument was solved by the neat assertion that success rates based on targets based on a choice of responses (including things like mammograms) of a therapy should or should not be used solely by RHEAVI patients.\newline%
This, most of us agree, is nonsense.\newline%

%


\begin{figure}[h!]%
\centering%
\includegraphics[width=120px]{./photos_from_epoch_8/samples_8_263.png}%
\caption{a man and a woman posing for a picture .}%
\end{figure}

%
\end{document}