\documentclass{article}%
\usepackage[T1]{fontenc}%
\usepackage[utf8]{inputenc}%
\usepackage{lmodern}%
\usepackage{textcomp}%
\usepackage{lastpage}%
\usepackage{graphicx}%
%
\title{proliferation and differentiation of i{-}ADSCsin neuronal cell}%
\author{\textit{Wang Zhu}}%
\date{03-04-2007}%
%
\begin{document}%
\normalsize%
\maketitle%
\section{The development of i{-}ADSCsin neuronal cell forms through an extended application of cellular environments to detect problems in central nervous system signaling processes that are enabled by natural differentiation has a profound impact on health and its applications}%
\label{sec:Thedevelopmentofi{-}ADSCsinneuronalcellformsthroughanextendedapplicationofcellularenvironmentstodetectproblemsincentralnervoussystemsignalingprocessesthatareenabledbynaturaldifferentiationhasaprofoundimpactonhealthanditsapplications}%
The development of i{-}ADSCsin neuronal cell forms through an extended application of cellular environments to detect problems in central nervous system signaling processes that are enabled by natural differentiation has a profound impact on health and its applications. This work was undertaken by the Distinguished Lecture of the National Convener of the Department of Neuroscience, Rosalind Fauntleroy (2007), with collaborators from the University of Groningen (NUEG), Norway and the University of Lund (Manau).\newline%
Neural cells play a key role in preparing our cells to produce functional proteins. However, their role in regulating our cell temperature and regulating the transmission of signals from cells in humans to other cells in primates raises ethical concerns. Given that the antibody market runs across the globe, our current efforts on creating the ability to control the cells to selectively target problems in central nervous system signaling are ineffective. The view that a good future may be less dependent on inhibition of differentiation can only be found by understanding the cell culture environment, or the amount of neuronal cells in neuron niches that are able to produce functional proteins in these niches.\newline%
“The development of i{-}ADSCsin neuronal cell forms through an extended application of cellular environments to detect problems in central nervous system signaling processes that are enabled by natural differentiation has a profound impact on health and its applications,” said Walter Lautersee, Norwegian Nobel Laureate and Dean of the ETH ETH Zurich School of Medicine, one of the leading scientists in the world. “Our study gives increasing insight into the role of a variety of protein niches in regulating neuronal cell temperatures as well as finding a way to develop regulatory factors that can make the cells to produce functional proteins. It is becoming increasingly clear that addressing these problems is only possible in monoclonal antibody medicine and ‘non{-}deo stimulation’ modalities.”\newline%
The involvement of neuroinformatics in its diverse research experiments\newline%
The presentation of the research was organized by Prof. Anand Gavrielle (UNSW) and Professor Chris Neftler (Fischwasth University in Groningen). In this event Prof. Gavrielle devoted much of the time in the examining of the scientific literature to discussing the importance of the aspect of the study of the development of i{-}ADSCsin neuronal cell forms in biology.\newline%

%


\begin{figure}[h!]%
\centering%
\includegraphics[width=120px]{./photos_from_epoch_8/samples_8_7.png}%
\caption{a man in a suit and tie is smiling .}%
\end{figure}

%
\end{document}