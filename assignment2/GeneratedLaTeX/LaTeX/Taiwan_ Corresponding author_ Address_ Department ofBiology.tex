\documentclass{article}%
\usepackage[T1]{fontenc}%
\usepackage[utf8]{inputenc}%
\usepackage{lmodern}%
\usepackage{textcomp}%
\usepackage{lastpage}%
\usepackage{graphicx}%
%
\title{Taiwan\_* Corresponding author\_ Address\_ Department ofBiology}%
\author{\textit{Chien Ling}}%
\date{04-04-2009}%
%
\begin{document}%
\normalsize%
\maketitle%
\section{These names have come up since I started covering Taiwan, giving cover for a major scientific scandal in Taiwan}%
\label{sec:ThesenameshavecomeupsinceIstartedcoveringTaiwan,givingcoverforamajorscientificscandalinTaiwan}%
These names have come up since I started covering Taiwan, giving cover for a major scientific scandal in Taiwan. The government is in a crisis that has wiped out most academic careers in the country of some 1.1 million people. Despite being the largest economy in Asia, it is a hard{-}hitting country to speak of. This is in the personal interest of one person, not only in the nation but also for Taiwan’s unique identity. I want to say nothing at all about the damage that is to be done by the government in the matter of Taiwan. Most of my investigations have focused on the performance of the Ministry of IUCN, now the Chinese military. This independence movement has been building for many years, but was stirred by the disputed territory in the northeast Asian states, where the resistance of the Chinese military has been rife and the controversial 1990s Chinese military takeover of Taiwan’s self{-}ruled island of Vanuatu, in which martial law was imposed.\newline%
Even though it seems to be controlled by the powerful Chinese regime, as my research shows, there is a global disconnect between the danger of confrontation and the potential to develop an escalating battle against China. This has been established in the latest history of “disproportionate” resource taking nations across the world through the “liberated territories” and “the naval reach” of Asia. The reason behind this is that the Chinese won’t recognize this boundary for the purpose of military drills. The reason also is that it is possible that a parallel warship might attack a continental Taiwan, if the mainland’s military force were to defy its territorial domination.\newline%
Clearly, an attack against China will be significant. To the extent that it causes casualties, this will lead to China, which will have to mobilize, and a large population to evacuate, thereby exposing the Chinese military to attack and this can create tremendous risks for our country. There is also a possibility that the Government will resist legitimate protests by political parties and the pro{-}democracy movements, and may even invite civilian demonstrations to counter the Chinese military action. This could just simply create confusion about who controls the mainland, and who should get military support to secure Taiwan. The fact that no one is talking about it publicly is unfortunate.\newline%
No one in Taiwan could speak of official neutrality when there is a danger of confrontation with China, but the situation has re{-}opened. As we know this is a tense diplomatic and strategic situation and an escalation of tensions may be inevitable. The country has a strong interest in a friendly foreign policy. If we are going to change Taiwan’s strategic position as a result of the Chinese military’s takeover of Taiwan, we must make the will for change seem clearly legitimate. In this aspect, politics is something that must be managed within the context of the whole Taiwan issue. So far, what has been found is that decisions taken by the Government of Taiwan and the Chinese Government have many sides.\newline%
The latest “blueprint” in Taiwan, which addresses both civilian and military concerns, has been released. It identifies several strategic threats being developed by the Chinese, including the military buildup to mainland China; a surge in Chinese military technology to the North Chinese People’s Liberation Army’s Long March and Shangri{-}La Dialogue symposium; and the continuation of the Shanghai Cooperation Organization’s pro{-}Western agenda.\newline%
The Beijing Declaration can be read as a cover for a new broad{-}based consensus on Taiwan. Rather than bickering between the Chinese Government and Taiwan’s power players about whether or not the request to put an air force unit in Taipei should be respected or over whose was the job, we should look to the other party. The Taipei Declaration will also be accompanied by President Chen Shui{-}bian’s proposed Constitution to end tyranny in the land that has not seen as much repression by the Chinese military.\newline%
The Law on Unlawful Army Deployment through the System of Defense and Order Resolutions is approved by the People’s Assembly. It preaches the need for public order to lead peace and harmony in the world.\newline%
When discussing Taiwan’s diplomacy within the multilateral community, this declaration overlooks the international standing of Taiwan. Taiwan’s position as a tolerant democracy has previously been dealt with by the Australian, the Indian, France and Germany, which have all closed their military bases in Taiwan.\newline%
In the past, the Taiwan people’s government has shown a serious commitment to maintaining peace and regional peace. I will never forget the moment President Chen Guangcheng came to see President Yu Kun, who wanted a dialogue to end a civil war and embrace compromise. Despite being not consulted in any way, President Chen appreciated President Yu

%


\begin{figure}[h!]%
\centering%
\includegraphics[width=120px]{./photos_from_epoch_8/samples_8_68.png}%
\caption{a woman in a red shirt and a black tie}%
\end{figure}

%
\end{document}