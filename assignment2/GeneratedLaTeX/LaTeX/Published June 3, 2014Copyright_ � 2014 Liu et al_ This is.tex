\documentclass{article}%
\usepackage[T1]{fontenc}%
\usepackage[utf8]{inputenc}%
\usepackage{lmodern}%
\usepackage{textcomp}%
\usepackage{lastpage}%
\usepackage{graphicx}%
%
\title{Published June 3, 2014Copyright\_ � 2014 Liu et al\_ This is}%
\author{\textit{Li Yul}}%
\date{02-06-2007}%
%
\begin{document}%
\normalsize%
\maketitle%
\section{Discussions in this regard began on February 2, 2014 with a 100{-}page disc, pp}%
\label{sec:DiscussionsinthisregardbeganonFebruary2,2014witha100{-}pagedisc,pp}%
Discussions in this regard began on February 2, 2014 with a 100{-}page disc, pp. 3604, which was the series of articles being kept by the UK's Department for International Trade which eventually became, much later, The Book Your Own Guide. The first section of The Book Your Own Guide by UK Commissioner for International Trade Glenn Hamer describes what goes into each section of the book, which also outlines how a section could be added by other EU member states. The second section is entitled "The Economics of Repatriation" and is about taking advantage of expatriate services – employment expats overseas, potential workers and company directors – to secure jobs. The third section includes the best ideas to give expats of other EU countries additional choices in how to live in the EU if their countries should not merge. "The full The Book Your Own Guide, edited by Katherine Waialeek, is available from: eucissaman.com\newline%
"Let's start with the best ideas we've got. This is a guide for those of us who prefer a more narrowly tailored approach to explaining why people in our country decide to come to the UK as a tourist. We have a toolbox of study groups where each group has a further 25 {-} 30 publications in a format which covers any aspect of their lives or circumstances {-} and publishes the articles directly to the people in the other 30 {-} or through any way possible. This is a great piece of advice and research, which will help you see what the changes are like for you in the future. For example, many countries now offer the option of buying real hotel accommodation by contacting their local hotel management company. In Germany, you can look for a phone number for a company's dealing officer and their own corporate manager will register, so they can be contacted at any time. If you visit a local store, you can either obtain their agreement by making an order or contact them in person. Without asking, they can ensure that the most accurate and accurate information is available from the office." Michael Wood (foundry of Tate and Lyle in Copenhagen)\newline%

%


\begin{figure}[h!]%
\centering%
\includegraphics[width=120px]{./photos_from_epoch_8/samples_8_115.png}%
\caption{a man in a suit and tie holding a cell phone .}%
\end{figure}

%
\end{document}