\documentclass{article}%
\usepackage[T1]{fontenc}%
\usepackage[utf8]{inputenc}%
\usepackage{lmodern}%
\usepackage{textcomp}%
\usepackage{lastpage}%
\usepackage{graphicx}%
%
\title{HIV{-}1 Tat Protein Increases Microglial Outward K+ Current and Resultant Neurotoxic Activity}%
\author{\textit{Fry Spencer}}%
\date{06-19-2001}%
%
\begin{document}%
\normalsize%
\maketitle%
\section{June 19, 2001 — In a study published in the journal New Drug Applications (NDA), researchers have found that HIV{-}1 Tat Protein increases activity in the brain {-} and has even suggested that it may help with the treatment of severe inflammatory bowel disease (IBD) by reducing the need for drugs to disrupt the inflammatory response}%
\label{sec:June19,2001InastudypublishedinthejournalNewDrugApplications(NDA),researchershavefoundthatHIV{-}1TatProteinincreasesactivityinthebrain{-}andhasevensuggestedthatitmayhelpwiththetreatmentofsevereinflammatoryboweldisease(IBD)byreducingtheneedfordrugstodisrupttheinflammatoryresponse}%
June 19, 2001 — In a study published in the journal New Drug Applications (NDA), researchers have found that HIV{-}1 Tat Protein increases activity in the brain {-} and has even suggested that it may help with the treatment of severe inflammatory bowel disease (IBD) by reducing the need for drugs to disrupt the inflammatory response. As reported in the April issue of NDA (NARA) from the American Translational Science and Engineering Research Institute (ATSERI), so far, the findings suggest that HIV{-}1 Tat Protein also increases synaptic transmission between the external and internal networks. This is important because HIV{-}1 cells in the bladder lose the ability to repair connections between cells and that will not be repaired if they are infected with HIV. Essentially, HIV has the capacity to fight against disease, so it usually depends on the HIV{-}1 metabolism for initiating HIV{-}1 metabolism. However, inhibitors of HIV{-}1 Tat Protein, such as CD4{-}albumin, have produced evidence that antiretroviral therapy may slow and halt HIV{-}1 progression.\newline%
In their study, Celine Davies and Jeanne Talaga of ATSERI and at Part 102 of the National Institutes of Health (NIH), have measured 717 HIV{-}1 Tat Protein ex vivo in the ATSERI brain. After eight months, the researchers found that the drive volume increased by more than 1,000 per cent in these rodents. Eleven months after one year, the activities remained relatively constant. In addition, fifteen distinct effects of exposure to HIV{-}1 have been observed in ATSERI{-}supported mice, with many promising.\newline%
“These findings demonstrate that both HIV{-}1 Tat Protein and CD4{-}albumin inhibit the signaling pathway signaling cells for the neurotic steps of the immune system, such as switching normal HIV/AIDS responses to natural responses,” says the senior author of the NDA paper, Stephen Bennett, Ph.D., associate professor of pharmacology and biochemistry at ATSERI. “Importantly, we see just the opposite, as synaptic evolution can carry greater influence on enhanced synaptic endurance in the brain, providing additional protection for the patient after HIV infection.”\newline%
In addition, the observation of this activity will play a significant role in future studies on how HIV{-}1 Tat Protein neutralizes either virus{-}1 or anti{-}infected molecules. Furthermore, the study suggests that HIV{-}1 Tat Protein provides a target for therapeutic response even in severe types of disease. The study found that HIV{-}1 Tat Protein also decreases antibodies stored in the liver or brain in rodents that suffer from Asperger’s syndrome, which concerns kidney conditions. Although HIV{-}1 Tat Protein has shown higher levels of activity in the brain in an experimental ATSERI{-}sponsored study that followed HIV{-}1 Tsubitor (CAT), CAT tests are less stable than their predecessors.\newline%
Recently, Discovery Science reported that a study of chronic HIV infection demonstrated that other drugs such as Tareepa have superior efficacy, although data is incomplete on the overall incidence of HIV{-}1 K+ pathology in the liver. As reported in the Eureka in 2010 issue of NDA (NARA), virus{-}1 mutation mutations in humans offer new insight into the proteins associated with inflammatory bowel disease, in which all HIV{-}1 cells are incorporated into the blood stream.\newline%
“HIV{-}1 Tat Protein provides sustained protection from disease and depression in these older brains, in which we normally associate the HIV{-}1 Nutritional Trogant gene with a discrete zone of open tissue in the liver,” says Celine Davies. “We are further planning to investigate how liver{-}specific depletion in HIV{-}1 Tat Protein can potentially replace fission proteins in non{-}HIV brain regions.”\newline%

%


\begin{figure}[h!]%
\centering%
\includegraphics[width=120px]{./photos_from_epoch_8/samples_8_281.png}%
\caption{a man wearing a hat and a hat .}%
\end{figure}

%
\end{document}