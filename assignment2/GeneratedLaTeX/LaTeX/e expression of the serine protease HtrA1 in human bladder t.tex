\documentclass{article}%
\usepackage[T1]{fontenc}%
\usepackage[utf8]{inputenc}%
\usepackage{lmodern}%
\usepackage{textcomp}%
\usepackage{lastpage}%
\usepackage{graphicx}%
%
\title{e expression of the serine protease HtrA1 in human bladder t}%
\author{\textit{Hsiung Yi}}%
\date{01-09-2006}%
%
\begin{document}%
\normalsize%
\maketitle%
\section{Researchers from the South Africa National Institute of Health have discovered a surprisingly significant hereditary contribution to urinary incontinence in adult women}%
\label{sec:ResearchersfromtheSouthAfricaNationalInstituteofHealthhavediscoveredasurprisinglysignificanthereditarycontributiontourinaryincontinenceinadultwomen}%
Researchers from the South Africa National Institute of Health have discovered a surprisingly significant hereditary contribution to urinary incontinence in adult women. As a result, the urine of men with a greater number of suspected urinary incontinence has a significantly greater than normal return of the urinary tract.\newline%
Thanks to a discovery made in Switzerland by the South African Human Capital Corporation in collaboration with the South African High Court, the urine of women with urinary incontinence returned to those with normal risk of recurrence, the study found.\newline%
The findings suggest that urinary incontinence is a possible cause of the common incontinence in male and female partners, and may trigger the development of a massive range of health problems in men in which a majority of partners suffer from incontinence.\newline%
The study’s authors noted the effect of the woman’s body contents was at the very start of the patient’s cycle of routine female continence, and the presence of the female bladder may have led to the occurrence of impairment of health.\newline%
“Citing the expression of the serine protease HtrA1 in adult women, and having detailed over 13 years of studies, we have established a significant molecular and practical challenge for this evaluation. More needs to be done to uncover this link,” says Professor Brian Willmers, Principal Investigator at SCINDI and Professor of Medicine at the Centre for Translational Medicine at the National Institute of Health (S\&H) in Bern.\newline%
New research out of SCINDI suggests that several health conditions can also have a large number of randomised, non{-}randomised, trial trial{-}driven morbidity and mortality studies. The author of the study, Professor Ian Smyth, says that this means, “this may have implications in the future for disease and health outcomes.”\newline%
Although the double{-}blind clinical trial was successfully conducted in 2002 to test urinary incontinence in women for 25 years, it was not implemented after the mid 2000s.\newline%
Although this should not be the final conclusion, the key issue was that both under a controlled transfer and the clinical trial had adverse associations that did not warrant further investigation.\newline%
Now, the SCINDI team is looking for additional data in order to evaluate its next steps.\newline%
Article: The home of the serine protease HtrA1 interlinking antigen, comments Francesca Gealey, Professor of Medicine at SCINDI, and Rachel Diesser, Chair of the SCINDI team, Redlands, CA, doi: 10.1097/suafep.1358804, published 17 January 2006.\newline%

%


\begin{figure}[h!]%
\centering%
\includegraphics[width=120px]{./photos_from_epoch_8/samples_8_480.png}%
\caption{a man in a suit and tie is smiling}%
\end{figure}

%
\end{document}