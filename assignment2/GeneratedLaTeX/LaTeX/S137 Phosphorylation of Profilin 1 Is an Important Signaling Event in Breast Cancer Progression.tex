\documentclass{article}%
\usepackage[T1]{fontenc}%
\usepackage[utf8]{inputenc}%
\usepackage{lmodern}%
\usepackage{textcomp}%
\usepackage{lastpage}%
\usepackage{graphicx}%
%
\title{S137 Phosphorylation of Profilin 1 Is an Important Signaling Event in Breast Cancer Progression}%
\author{\textit{Glover Nicole}}%
\date{01-11-1997}%
%
\begin{document}%
\normalsize%
\maketitle%
\section{NEW YORK, N}%
\label{sec:NEWYORK,N}%
NEW YORK, N.Y., Jan. 11, 1997 – The site of the Zathura Adolescent Cancer Center, Cayespena, is currently being analyzed to determine the exact health of the breast cancer patients and their families. The results of the study revealed results that have a favorable overall prognosis for both the immediate victims and the future patients. “Although breast cancer is presumed to be among the most common diseases in the general population, breast cancer comes in several classes,” concludes Dr. William Rosenberg, principal investigator of the Zathura Adolescent Cancer Center’s Center of Excellence, where the study was conducted. “Unlike in vitro fertilization or anti{-}fertility therapies, chemotherapy and prognosis reports from a state, and clinical protocols, the patients were treated on their own with all necessary information in a systematic way”.\newline%
To conduct the study, the scientists looked at the medical record of 27,557 K{-}12 subjects at the NYU Center for Cancer and Blood Disorders. The subjects were seen prior to surgery for breast cancer. The following question was asked:\newline%
— Did the patient ever undergo surgery to correct a mutation of the BRCA1 and BRCA2 genes?\newline%
— Was breast cancer diagnosed by the same person once that their 5{-}year prognosis had improved?\newline%
— Was the diagnosis based on a radiological screening test or imaging, and did the nurse nurse check out the patients to determine the diagnosis?\newline%
— Was the diagnosis recorded by a stool sample?\newline%
— Was the hospital screening done either by a physician assistant, a doctor, or an administrator?\newline%
— Did the treatment first transpire first in the ER?\newline%
— Was the treatment done on the skin or the other way around?\newline%
— Did the treatment be characterized as therapy for the diagnosis?\newline%
— Did the treatment be guided?\newline%
The Zathura Adolescent Cancer Center will study the results of its post{-}treatment studies to determine whether treatment for this disease or for the next two to three years will be successful for both the victim and patient. “It’s a definite indicator of cancer treatment being successful for the patient as compared to the other options,” reports Dr. Rosenberg. “The atypical way to beat this cancer is by pre{-}emptive surgery. People have been encouraged to do house{-}building while waiting for what happens.”\newline%
The findings are being presented at an upcoming session of the American Society of Clinical Oncology (ASCO) annual conference in San Francisco.\newline%
For the study, the Zathura Adolescent Cancer Center and Cayespena were performed by the Research and Engineering Center of Syracuse University. The four{-}year longitudinal study is held every ten years and is affiliated with the Center for Fumarate{-}Bio{-}Records, the Center for Electrocortia Research, and the Case Study of Breast Cancer (CIRRC) Institute. The results of this study were released last summer in The Journal of the American Medical Association.\newline%
The Zathura Adolescent Cancer Center receives only 600 patients annually through its establishment of the JCCV02 program to prepare two pre{-}treatment studies per year. The core research is conducted from the Zathura Adolescent Cancer Center’s Campus in New York City, Syracuse, NY.\newline%
The Zathura Adolescent Cancer Center’s special focus this year is to investigate the effects of hormone therapy on breast cancer survivors. After current women undergo all the necessary therapy, a prognosis has already been formulated for the subsequent years after surgery in which tissue could be removed. “The goal is to determine how the prognosis will change after treatment of this disease,” Dr. Rosenberg adds. “This is a major milestone in the center’s history of raising awareness of the many symptoms of breast cancer.” He further estimates that over 30 percent of patients will experience debilitating symptoms, including chest discomfort, painful labia, internal bleeding, weight loss, fatigue, fatigue, persistent, and no need for surgery.\newline%
The Zathura Adolescent Cancer Center is committed to advancing interdisciplinary research in rare diseases and cancers by focusing on treatments, labs, and public health. “This research organization has been a tremendous supporter of the Center,” Ms. Haemath Vora, The Zathura Adolescent Cancer Center’s Physician Professor at NYU School of Medicine, said. “From emotional support to practical advice on how best to treat pre{-}cancerous lesions, these efforts have led to a great deal of medical success for patients”.\newline%
Source: S.A. Chelsea Huang\newline%

%


\begin{figure}[h!]%
\centering%
\includegraphics[width=120px]{./photos_from_epoch_8/samples_8_354.png}%
\caption{a man with a beard and a tie}%
\end{figure}

%
\end{document}