\documentclass{article}%
\usepackage[T1]{fontenc}%
\usepackage[utf8]{inputenc}%
\usepackage{lmodern}%
\usepackage{textcomp}%
\usepackage{lastpage}%
\usepackage{graphicx}%
%
\title{www\_fapesp\_br\_)\_ The funders had no role in study design, da}%
\author{\textit{Tao Lee}}%
\date{05-28-2006}%
%
\begin{document}%
\normalsize%
\maketitle%
\section{by L}%
\label{sec:byL}%
by L. Dawson | Technology: keep up with our technological progress in 2009. Edited by Fawn Balcomb\newline%
Ten projects – particularly those focused on global business – presented at the Forum on Technology in Berlin last week represent the beginning of a new chapter in the evolution of technology.\newline%
The exchange between the communications, information and technology industries, workshops and the world’s leading trade and information professionals had been going on for several years, but was mostly dormant until last year.\newline%
As part of a new Digital Business Summit, the inaugural Forum on Technology in Berlin (FBC), a worldwide event showcasing successful B2B services and firms was put on as a showcase event.\newline%
The speakers were:\newline%
• Costengira Barbara Mathieu\newline%
• Venture Europe\& Society, Alexandra Boscombe\newline%
• Peter Blackburn\newline%
• Design, development and engineering, Fred Gaym,\newline%
• The Knowledge Market and Technology Hub\newline%
• Marcia Goins, GISP.\newline%
“There was no regard for the digital arts and life sciences in Berlin last year,” says the FBC’s co{-}ordinator, Professor Alice Taaffe, a specialist in B2B innovation from Shende, Germany.\newline%
“It would have been impressive if this was a regular event, but in the end FBC kept on saying no.”\newline%
Says Ms Taaffe: “If this had been a regular event, we would have been amazed by the amount of speakers and communication in Berlin.”\newline%
When this year’s Forum on Technology began it was only after it had a shortlist of 10 representatives for such events.\newline%
FBC held three business events in Berlin including Web.co.za Design in the Conference Hall on May 14, A4S on May 14 and GVE on May 14.\newline%
A reality check\newline%
Despite the conference’s credentials, it had a problem of speed. “As we were speaking, we were discussing the view of the relevant information on our favoured media areas,” says Ms Taaffe.\newline%
Further, the lack of to{-}do list led to a picture of a poorly organised organisation when it came to finding the right topics to cover for the business groups’ pressing issues.\newline%
No one had the notes and lists, nor was there any proper advice available to the speakers.\newline%
The difference this year is that this time the business focus has been decided by group sponsors, including the Association of German Cities and the National Confederation of Society with Major Media.\newline%
They wanted delegates to show their expertise in electronic communication and business development and were happy with the focus given to the theme for 2009.\newline%
The key, of course, is to build a relationship between management, inventors and business experts.\newline%
The Forum proved to be a fruitful and productive collaboration.\newline%
Trade\newline%
“One of the key messages of the Forum was that there has been a change in the skill{-}based approach to the sector. It was remarkable that technical experts were interested in communicating how efficient they were working, but with the revision of things we’re seeing in the economy,” says Ms Taaffe.\newline%
“In every business, everyone works better by working together rather than distorting things.”\newline%
Despite having some expert speakers at the Forum, the event wasn’t without criticism. “One of the major challenges was that the industry representatives didn’t participate in the dialogue with each other,” says Ms Taaffe.\newline%
“The Forum didn’t understand whether they would have been very happy with which format was being used.\newline%
“We created a forum for the development of a ‘The best hybrid’ approach and the idea of a ‘Summer Programme’ was accepted but because it had no share in the public realm, it got extremely mixed response.”\newline%
Another not so pleasant experience was the need to organise workshops in the works and studio space – so that both participants could get feedback on what their respective ideas were like.\newline%
“That was a huge criticism,” says Ms Taaffe. “The only areas we were able to explore were new concepts and initial ideas, rather than conventional methods.”\newline%
FBC is looking forward to doing more on the future of enterprise communication as an industry “perseverance” is sought.\newline%
Paul Calandra, deputy director, media, has been working on the launch of future B2B research and collaboration since 2008, and has already started small development projects for the sector.\newline%
“We’re on track,” he says. “We are working towards the transition into the B2B sector with a cluster of new products and services to be introduced at the end of

%


\begin{figure}[h!]%
\centering%
\includegraphics[width=120px]{./photos_from_epoch_8/samples_8_217.png}%
\caption{a man and woman pose for a picture .}%
\end{figure}

%
\end{document}