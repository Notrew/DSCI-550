\documentclass{article}%
\usepackage[T1]{fontenc}%
\usepackage[utf8]{inputenc}%
\usepackage{lmodern}%
\usepackage{textcomp}%
\usepackage{lastpage}%
\usepackage{graphicx}%
%
\title{ript\_ available in PMC 2014 July 13\_Published in final edite}%
\author{\textit{Sun Shen}}%
\date{04-01-2002}%
%
\begin{document}%
\normalsize%
\maketitle%
\section{Parliamentary Quarterly Note Report of the fiscal year 2001 {-} 2003 , seen from the finance minister’s desk}%
\label{sec:ParliamentaryQuarterlyNoteReportofthefiscalyear2001{-}2003,seenfromthefinanceministersdesk}%
Parliamentary Quarterly Note Report of the fiscal year 2001 {-} 2003 , seen from the finance minister’s desk.\newline%
1. 7.4 4.0 10.4 27.7 16.8 2.3\newline%
Let me take a stab at a macroeconomic one from the recent report on the fiscal year 2001–2003 and take this for the purpose as follows:\newline%
1. Revenue generation rose from 5.8 billion to 5.5 billion in the year 2001–03. It increased 5 percent, this and other things, up to 1.4 billion.\newline%
(2) General income (a measure of health{-}ness’s appeal to the average citizen, i.e. the average citizen’s overall health) rose from 4.2 billion to 4.8 billion. This rise was largely due to 2.9 billion higher GST revenue collections.\newline%
(3) An additional 0.9 billion went to the energy and utilities budget, partly to be replenished from H12 2001, except in Pakistan.\newline%
(4) Despite losing seats to a backdrop of deteriorating economic conditions, 14 percent of party faithful voted in the Parliament’s Sectoral Land Committee votes.\newline%
“Social service expenditure expenditure rose 9 percent year on year,” explains Mr {[}Spence{]}, “and also saw a 14 percent jump in overall welfare expenditure after the election.”\newline%
2. 4.4 6.0 14.6 9.8 3.8\newline%
(5) Household contribution: in 2002 budget (24.9 billion) , more than 14 percent of National Assembly citizens contributed to government. Now, this expenditure rate is very high, a response to the unprecedented level of corruption which the Ministry of Planning and Development has suffered.\newline%
2. Private Finance: in 2002, household contribution was at 6.9 billion while of this 7.3 billion came from private companies, although this increased from 7.9 billion in 2002 to 9.9 billion in 2003.\newline%
3. Housing: in 2002, the average house had a value of 8.7 bn, down 15 percent from 2001/03. This is mainly due to the declining level of private investment, very low returns on non{-}poor investment (average home{-}buyer turnover: 1.5 billion) and concern over rising property prices (average new build cost: 7.3 billion).\newline%
(4) Revenues: in 2002, the total revenue reached 77.9 billion {-} up from 83.8 billion. The election gained significant level of revenues through the large part of the spending during the year, I’m very happy with this. Although most industries were hit by fluctuations, none suffered major losses.\newline%

%


\begin{figure}[h!]%
\centering%
\includegraphics[width=120px]{./photos_from_epoch_8/samples_8_116.png}%
\caption{a man in a suit and tie holding a wii controller .}%
\end{figure}

%
\end{document}