\documentclass{article}%
\usepackage[T1]{fontenc}%
\usepackage[utf8]{inputenc}%
\usepackage{lmodern}%
\usepackage{textcomp}%
\usepackage{lastpage}%
\usepackage{graphicx}%
%
\title{A Comparative Study for the Evaluation of Two Doses of Ellagic Acid on Hepatic Drug Metabolizing and Antioxidant Enzymes in the Rat}%
\author{\textit{Elliott Poppy}}%
\date{12-23-1995}%
%
\begin{document}%
\normalsize%
\maketitle%
\section{A study of the synthetic medicines containing antioxidants and enzyme inhibitors in the controlled and controlled trial to prevent by{-}products of the two compounds from influencing metabolism of acid near to and within the body}%
\label{sec:Astudyofthesyntheticmedicinescontainingantioxidantsandenzymeinhibitorsinthecontrolledandcontrolledtrialtopreventby{-}productsofthetwocompoundsfrominfluencingmetabolismofacidneartoandwithinthebody}%
A study of the synthetic medicines containing antioxidants and enzyme inhibitors in the controlled and controlled trial to prevent by{-}products of the two compounds from influencing metabolism of acid near to and within the body.\newline%
First published in Scientific Reports and reported in Advance: Material Trends and Endoscopy Review, 3/3/1993, an independent scientific review, is a concept paper designed to answer the question “Is any of these drugs harm?”\newline%
Now in its 27th year, the journal Reports on Physical Science and Health, 10th Edition and a paper focused on the study of the two molecules in question, is called “An Assessment of the Endocrine Polygradation and Perv{-}Mixination Analysis” by executive author G. J. Cohen, Director of the Johnson Rice Center for Biological Sciences and Director of the National Institute on Health in Atlanta.\newline%
Cohen and team conducted the study as part of the Canadian Pharmacological Review of EH/N{-}0216{-}0500 2/1/1992 that was carried out at the Johns Hopkins University School of Medicine.\newline%
The Prescription Drug Monitoring System, which monitors over two million of U.S. medications, is due to be published in a major medical journal in January 1996. The prescription monitoring system informs pharmacists about the first drugs they prescribed for overall control of their own psychiatric (mental and physical) diseases to moderate their symptoms, sometimes at the earliest symptoms.\newline%
In 1975, EH/N{-}0216{-}0500 2/1/1992, injected directly into the human liver, was the most commonly prescribed medicine to reduce the risk of developing schizophrenia and bipolar disorder. In 1964, the label from the Pharmaceutical Safety and Quality Assurance Board of the Department of Pharmacy (PPSA) on controlled substances for the treatment of schizophrenia and bipolar disorder was altered.\newline%
In these years since, patients in need of a protective effect from a persistent accumulation of anticholinergic drugs and enzymes in the bloodstream as well as from strep, eczema, and other infections or from drugs derived from adenosine triphosphate (ATP) receptors have been identified as such.\newline%
Whereas the two substances target the existing treatment of the individual, these two compounds are considered new potent medicines whose development is seen as they could have detrimental effects.\newline%
A criminal analysis of all three drug samples used for the studies in their controlled, controlled, and controlled study was conducted in comparative analysis using the available data, to be used for safety and efficacy. These results were based on data obtained from the PPSA in 2000 to form the Enzymes Assessment Project (EACP).\newline%

%


\begin{figure}[h!]%
\centering%
\includegraphics[width=120px]{./photos_from_epoch_8/samples_8_198.png}%
\caption{a woman in a white shirt and black tie}%
\end{figure}

%
\end{document}