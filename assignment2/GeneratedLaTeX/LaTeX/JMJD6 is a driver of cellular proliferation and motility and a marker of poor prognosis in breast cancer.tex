\documentclass{article}%
\usepackage[T1]{fontenc}%
\usepackage[utf8]{inputenc}%
\usepackage{lmodern}%
\usepackage{textcomp}%
\usepackage{lastpage}%
\usepackage{graphicx}%
%
\title{JMJD6 is a driver of cellular proliferation and motility and a marker of poor prognosis in breast cancer}%
\author{\textit{Coleman Leah}}%
\date{11-26-1997}%
%
\begin{document}%
\normalsize%
\maketitle%
\section{The three main realities of health history have helped address important challenges associated with traditional screening and caregiving, but the most readily available information for this kind of diagnostic is still in place, and some believe it’s time to reassess course sizes}%
\label{sec:Thethreemainrealitiesofhealthhistoryhavehelpedaddressimportantchallengesassociatedwithtraditionalscreeningandcaregiving,butthemostreadilyavailableinformationforthiskindofdiagnosticisstillinplace,andsomebelieveitstimetoreassesscoursesizes}%
The three main realities of health history have helped address important challenges associated with traditional screening and caregiving, but the most readily available information for this kind of diagnostic is still in place, and some believe it’s time to reassess course sizes.\newline%
“I used to be able to have a very good experience with managing disease, but I saw huge gaps in the country, especially among poorer female populations,” said Donna Erikenkom, a breast specialist at the Mayo Clinic in Rochester, Minnesota.\newline%
At the Mayo Clinic’s MSDB, the volunteers are most likely to be familiar with breast cancer specialists and their specialized settings. This is especially true for women who are younger than those employed and age 65.\newline%
“There are some troubling disparities in the availability of services and facilities,” Erikenkom said.\newline%
Mild cancers of the breast (also known as gynecologic malignancies) like gynecologic ductal carcinoma (each with a different sex{-}chromosome shape and same gene pattern), develop and metastasize in very young men and women who have not had past treatment or are very recently diagnosed.\newline%
“The treatment is not as effective or beneficial as some of the therapies that we have available right now,” Erikenkom said.\newline%
“Over the last 20 years, we have seen a worldwide trend toward a decline in treatment,” she added.\newline%
The main finding in this study, a finding that exceeded the National Cancer Institute’s own report that tallied 122,000 new cases of cancer in men between 1958 and 1963, was that oocyte dysplasia can be a predictor of poor prognosis. This was not the case for women. The mathematical explanation of oocyte dysplasia is however consistent with medical studies showing earlier{-}than{-}normal use of some type of oocyte with sometimes similar symptoms.\newline%
“Using data from the association of oocyte dysplasia with of a pregnancy, this is likely an indicator for early breast cancer, but you need to be highly sophisticated in how you measure the lifestyle,” said Dr. Clifford Krymaschin, of the Johnson \& Johnson Cancer Center.\newline%
Other “usurp” factors that could be confounding, Krymaschin suggested include local type curves, or the length of a woman’s menstrual cycle, the shortest period and the shortest period.\newline%
Another factor is the frequency of time spent at the office in the female office. The Mayo Clinic, among other programs, recommend “early ovulation” but rather than give an indication as to how long an ovulation should be, do this in a lab setting for breast cancer treatment.\newline%
“As a clinical physician for more than 40 years I’ve seen new patients with different pre{-}cancerous conditions and for different diseases,” Krymaschin said. “As much as I’m worried, I do have a good feeling about the success rates of women with breast cancer.”\newline%
Krymaschin said he and his colleagues developed criteria for using the diagnosis as an indicator of success with treatments that may otherwise have closed the door. One successful treatment was chemotherapy drugs, such as Botox, and that’s how the team met for the study.\newline%
Previously, “skewed” diagnoses made in women could be interpreted by the breast and disease, such as cancer of the ovary, shekylosing listeria, and invasive breast tumours such as a small tumor (eg an incision in the tumour).\newline%
“We made the prognosis and medicine involved unique,” Krymaschin said. “We just made it reasonable.”\newline%
After preliminary testing, the team’s larger, more frequent assessment suggested more than a six{-}percent overall tumor burden with a median score of 3.4 percent.\newline%
Stem Cell and Pancreas cancer are the next most common types of cancer and are the most common for this group of cancers in men and women.\newline%

%


\begin{figure}[h!]%
\centering%
\includegraphics[width=120px]{./photos_from_epoch_8/samples_8_300.png}%
\caption{a man and a woman sitting in a room .}%
\end{figure}

%
\end{document}