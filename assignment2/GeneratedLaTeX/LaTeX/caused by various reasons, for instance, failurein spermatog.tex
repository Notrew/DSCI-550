\documentclass{article}%
\usepackage[T1]{fontenc}%
\usepackage[utf8]{inputenc}%
\usepackage{lmodern}%
\usepackage{textcomp}%
\usepackage{lastpage}%
\usepackage{graphicx}%
%
\title{caused by various reasons, for instance, failurein spermatog}%
\author{\textit{Hsiao Lili}}%
\date{05-18-2006}%
%
\begin{document}%
\normalsize%
\maketitle%
\section{The inability of the authorities to either rectify or avoid accusations against Minister Tom Flaherty is putting the country in a bad light – literally and figuratively}%
\label{sec:TheinabilityoftheauthoritiestoeitherrectifyoravoidaccusationsagainstMinisterTomFlahertyisputtingthecountryinabadlightliterallyandfiguratively}%
The inability of the authorities to either rectify or avoid accusations against Minister Tom Flaherty is putting the country in a bad light – literally and figuratively. Mr. Flaherty was a high{-}risk, moderate figure who was seen to be a threat to the government’s all important goals: achieving the Government’s reform agenda and getting its act together.\newline%
At the time he was given the opening by Foreign Affairs Minister Eric Hoskins to explain what had happened and what it would take to address. Mr. Flaherty was regarded as moderate and a moderate to blame, apparently because his approach to matters of national security has been portrayed too often. It was during his criticism of the Government’s handling of Al{-}Qaeda case for a British journalist whom he had interviewed, he also downplayed the use of his friendship with the journalist, Valerie Plame, to criticize the Government’s handling of Mr. Hoskins’ actions during their (alleged) discussions. He went on to point out that Mr. Hoskins’ office had no powers whatsoever to end a contact between the Prime Minister and the journalist, James Goldsmith. After the accounts of Mr. Hoskins’ meetings with Ms. Plame, all rights were denied to reporters. Mr. Flaherty revealed that this incident had made him even more agitated than before. He also revealed that he had ordered a review of the independent investigation into this incident.\newline%
Apart from the point that Mr. Flaherty is one of the most irresponsible officials in recent history, Mr. Hoskins could not have been more prescient. In September 2001, he commented that there were some who were “towards to nebulisation” of a reference to the early days of the McGrath inquiry, in which Robert “McClive” McNamara was appointed executive. To him, his reflection was more dangerous than Mr. Harper’s public comments in the weeks following the carnage on September 11, 2001. It is an incantatory bellwether of the commission’s findings, which Mr. Flaherty’s attacks, particularly the frequent use of last names, seem to represent. The reference appears to be used to describe the considerable security situation present in the country, especially in relation to Mr. Flaherty’s ability to return the world to normal after a security crisis has taken place.\newline%
In April of 2001, Mr. Flaherty was again invited to address an annual parliamentary session – at the same time, there were no protests or follow{-}up pressure to end the investigation which resulted in John Wilson being sentenced to 15 years in prison in 2006. His comments in response to the murder of Drummond Hill analyst Marcus Hogarth and the killing of Gregory Scott on June 8, 2001, in an incident which was the subject of a controversy, caught public attention and led to an apology from Mr. Flaherty. Mr. Flaherty also publicly apologized to the families of Mr. Hogarth and Scott and made a public apology to Mr. Scott during the National Council Forum at Parliament House, where Mr. Flaherty was seated. This incident was taken apart and looked at as a place to bring one to closure, rather than one to close a long and distinguished history. It left the proverbial question of open leadership to be put to rest.\newline%
Following that incident, there was a high level of criticism of Mr. Flaherty’s conduct, political opportunism and an expediency. The following month, there was a paper written by Margaret MacDonald of the British Conservative Party that was partially cleared up. Now the subject matter of the paper is being dealt with, by a member of Parliament, who has been under investigation for breaking party rules.\newline%
Three members of Parliament said they were also under investigation and allegations were made that we did not interview or question the Minister. At the same time, Mr. Flaherty claimed he had not met Mr. Wilson and that he had simply “used the words of the former official” to refer to him. Of course, the Minister is accused of failing to meet the terms of the watchbill, an action that the British Parliament was supposed to rectify. This matter fell under the Office of the Prime Minister’s spotlight and is no longer being investigated. Therefore, we see from the debate that the contempt for the integrity of the Parliament is becoming less and less routine.\newline%
The discourse in the country should begin with Mr. Flaherty’s brother.\newline%
Trevor Chamandy is the founder of the Arcs, the high{-}performance coaches of Ireland\newline%

%


\begin{figure}[h!]%
\centering%
\includegraphics[width=120px]{./photos_from_epoch_8/samples_8_286.png}%
\caption{a man with a beard and a tie is smiling .}%
\end{figure}

%
\end{document}