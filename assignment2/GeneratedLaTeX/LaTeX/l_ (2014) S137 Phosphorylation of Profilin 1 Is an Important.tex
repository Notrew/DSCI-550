\documentclass{article}%
\usepackage[T1]{fontenc}%
\usepackage[utf8]{inputenc}%
\usepackage{lmodern}%
\usepackage{textcomp}%
\usepackage{lastpage}%
\usepackage{graphicx}%
%
\title{l\_ (2014) S137 Phosphorylation of Profilin 1 Is an Important}%
\author{\textit{Chung Peng}}%
\date{05-10-2001}%
%
\begin{document}%
\normalsize%
\maketitle%
\section{I was prepared for the familiar: “Is that class fantastic?”\newline%
It was the slogan that fell forth to print across my retail catalog}%
\label{sec:IwaspreparedforthefamiliarIsthatclassfantastic?Itwasthesloganthatfellforthtoprintacrossmyretailcatalog}%
I was prepared for the familiar: “Is that class fantastic?”\newline%
It was the slogan that fell forth to print across my retail catalog. I had purchased a new T{-}shirt and a cute fluorescent orange outfit. We’d studied the revolution in materials that required new ways of remembering and organizing complex systems and materials.\newline%
I’d coped with this idea with work done in the form of randomized, randomized patient medical trials. But it also had the close relative text that somehow arrived at my heart that chimed with the reality of sequencing arrays.\newline%
A little while later, I learned that my theory of theoretical and statistical verification/patterns of causality had been proven correct. The college of physics was a benevolent repository for math and science. I wanted to study it.\newline%
I didn’t mope and wait for the next test to test to teach. Yet when we looked hard at results, I discovered that my hypothesis had legs. In a related consideration, recent advances in biomarkers analysis had yielded a pleasing “win{-}win” result. In many disciplines, these tools have been used by medical practitioners to explore subtle variations in a person’s personality and behaviors. They’ve also revolutionized medicine and propelled surgeon knowledge.\newline%
My work with psychosomatic sequencing, as a psychiatrist at the University of Leningrad’s (Rodent and Ordinary Disease) Hebrew College Hospital, has led me to one of the few educational institutions that actually offers psychosomatic samples to credentialed medical students.\newline%
They put these samples to work on the Gafnar’s Chesti Hämaaat on the Schroders Global Program for the National Down Syndrome Society and have already received about 170 of these samples. A part of the team is using them to explore sequencing related behaviors.\newline%
As well as greatly advancing my understanding of my behavior, I’m excited about the implications of this training for future medical medicine.\newline%
I can only imagine how seriously my colleagues and I are taking my research into this area. The materials and techniques contained in this class will make it possible to get the info to people around the world. I hope this knowledge will ultimately lead to a better outcome for people afflicted with Down syndrome.\newline%
David Epstein is a Ph.D. candidate at UCLA Medical Center who specializes in psychiatry and clinical psychology. He served on the National Clinical Committee on Down Syndrome for the American Psychological Association. He currently teaches psychosomatic sequencing and genetic susceptibility for Down syndrome, an important area for treatment. He is also a co{-}editor of the Autism Research Journal and holds an Abcy H.D. special mention in the fall 2001 edition of his journal as a Juvenile. He lives in Lauen, Norway with his wife.\newline%

%


\begin{figure}[h!]%
\centering%
\includegraphics[width=120px]{./photos_from_epoch_8/samples_8_248.png}%
\caption{a man in a suit and tie is smiling .}%
\end{figure}

%
\end{document}