\documentclass{article}%
\usepackage[T1]{fontenc}%
\usepackage[utf8]{inputenc}%
\usepackage{lmodern}%
\usepackage{textcomp}%
\usepackage{lastpage}%
\usepackage{graphicx}%
%
\title{espondence to\_ Luke McCaffrey Department of Oncology McGill}%
\author{\textit{Kung Li}}%
\date{11-13-1999}%
%
\begin{document}%
\normalsize%
\maketitle%
\section{It must be no surprise that the vast majority of patent applications are brought at the insistence of the typical patient at the heart of the medical community}%
\label{sec:Itmustbenosurprisethatthevastmajorityofpatentapplicationsarebroughtattheinsistenceofthetypicalpatientattheheartofthemedicalcommunity}%
It must be no surprise that the vast majority of patent applications are brought at the insistence of the typical patient at the heart of the medical community. This is a fact that both regulator CellCelex and the Care Standards Board (CSCB) have come to appreciate.\newline%
The stated goal of the CSCB in its initial report to the general public in 2004 was to "provide excellence in the provision of specialised information and consultation for the public". The report has guided the CSCB’s legal strategies to avoid unnecessarily dragging its feet on some or all of the crucial matters that must be resolved in the early stages of the telecommunications transaction.\newline%
In July, for example, the CSCB considered 15 of the 140 submissions received by CellCelex pertaining to matters relating to the convergence of cellular data, WiFi, in{-}network communications. Four of the recommendations, including an instruction for the regulator on the law and IT standards, are to be upheld, as are the legal advice, email correspondence and other information required for telecommunications communication. This advice was released in the course of the report.\newline%
Naturally, there were objections from one side of the stand{-}off between CellCelex and the CSCB, which included lawyer Arthur Forsold who claimed that the CSCB was merely following the law by not consulting with the CSCB. Another side accused the CSCB of diminishing its authority.\newline%
However, this resistance took hold soon after, in 2005, when CellCelex presented its application for a POTENTIAL regulatory order to the CSCB. The ruling required CellCelex to comply with a number of essential matters, including the interconnection agreement, the data{-}screening technology required for all communications, telemedicine and ePrescribing technologies, the provision of adequate safeguards to protect patients' privacy and basic information received by them; and the existence of a definitive baseline to gauge the evolving regulatory regime and complications associated with obtaining information.\newline%
This argument may not be as convincing as initially expected, since there were other submissions in 2005, which included a series of joint submission briefs, sent to the CSCB by the specialists and the regulator, and submissions to guidance bodies that CellCelex was prepared to support. But the most important aspect of the two cases is the significantly defined underlying principles of information and consultation.\newline%
In both cases, the principle was promoted by the CSCB as a useful tool in engaging with doctors and staff, because it suggests that communications must be exchanged to facilitate knowledge exchange. However, the CSCB is keen to rule that to facilitate this communication there must be an understanding of the nature of the communication between the specialist and doctor. This requirement, as laid down by the Administrative Procedure Act in 2001, is now being reconciled with the regulatory regime that CellCelex and the CSCB have now entered into. As such, to that end, the case must remain technical and conceptualised and developed in the view of both the CSCB and the CSCB.\newline%
This principle of information and consultation must now be communicated to the broader body of medical staff in the field so that the management of clinical information and communications might advance to date, without the need to rush to judgment. Without a communication flow into the field, there would be no less a number of problems for practice. From this perspective, the principle of information and consultation must be transferred to the same regular contact that sees patients come in regularly to a company where clinicians have decided what they want to communicate, in contrast to the longer{-}established practice that the CSCB believes is currently misleading.\newline%
We welcome the reasoned, highly reasoned and highly credible decision by the CSCB and the CSCB.\newline%
K. McCaffrey is Chief Executive Officer of CellCelex Laboratories.\newline%

%


\begin{figure}[h!]%
\centering%
\includegraphics[width=120px]{./photos_from_epoch_8/samples_8_202.png}%
\caption{a black and white photo of a man and a woman .}%
\end{figure}

%
\end{document}