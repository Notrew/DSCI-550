\documentclass{article}%
\usepackage[T1]{fontenc}%
\usepackage[utf8]{inputenc}%
\usepackage{lmodern}%
\usepackage{textcomp}%
\usepackage{lastpage}%
\usepackage{graphicx}%
%
\title{Meningococcal Porin PorB Prevents Cellular Apoptosis in a Toll{-}Like Receptor 2{-} and NF{-}B{-}Independent Manner}%
\author{\textit{Bruce Jacob}}%
\date{11-21-1993}%
%
\begin{document}%
\normalsize%
\maketitle%
\section{Two men show signs of the TLA before and after curing their toxic sleep{-}obesity{-}induced S{-}barge virus (Menedococcal), a disease transmitted by the nose, mouth and upper respiratory tract}%
\label{sec:TwomenshowsignsoftheTLAbeforeandaftercuringtheirtoxicsleep{-}obesity{-}inducedS{-}bargevirus(Menedococcal),adiseasetransmittedbythenose,mouthandupperrespiratorytract}%
Two men show signs of the TLA before and after curing their toxic sleep{-}obesity{-}induced S{-}barge virus (Menedococcal), a disease transmitted by the nose, mouth and upper respiratory tract. (B. A, c, o, b, ati, P, xpx) — A new study by Johns Hopkins University and the American Heart Association (ASHA) that investigated physiological and clinical characteristics of the men’s Meningococcal Porin (M{-}P) infections, emerged early and affected young women. The study is published online by PLoS ONE (Preliminary findings link to French Journal of Epidemiology and Community Health by Dax Piotralka and A. Garsalla).\newline%
These men are also researchers in Paul Naugle’s recently published University Journal of Clinical Infectious Diseases in which they first provided a summary study of the 30 men over a period of 10 years in Germany, France, and the United Kingdom. They then added Piotralka and A, and focused on the men’s expression of the TLA. Interviews were conducted with 11 lesions. On each of the lesions on the men’s DNA, they observed cells; cultured to see how the cells functioned. When the lesions stopped working, they showed declining tumor activity.\newline%
When these men were HIV{-}positive, it was not possible to measure the disease in their cell lines. One year later, they were well managed. The signs remained in these men for the next 6 months, with the appearance of still developing TLA at the serum level (as seen in these young men’s skin on the skin they shared) after eight months. They also show signs of increased activity at site of the disease.\newline%
One year later, three lesions on both tissue and flesh were replaced with progeria. This disease continued to progress, showing the presence of the long{-}term genetic mutation from prior attempts to enter a cell line in its path. The virulence pattern persisted the next nine months, with the mortality rate, compared to the control group, beginning with a doubling of exposure. The dose of cell{-}regulating drugs, including amoxicillin, (The Stimulant Stimulant), was reduced, although by more than 30 percent. The symptoms, like fatigue, changes in appearance, with surprising rates of sepsis and meningitis, were mild.\newline%
It was only after the men were HIV{-}positive that infections of the different transthyretin levels reduced, with the substance becoming more common in those who were diagnosed as positive, while those who were negative were not at all. The men’s immune response increased in the short{-}term, but the survival rate remained very poor for the young men.\newline%
All in all, the results showed that as the virus became less active and more normal, the proteins in the TLA on the skin began to appear, such that as the skin became more healthy, the proteins appeared to become thin as they entered the cells. By the fourth week of the infection, the TLA in the tissues had vanished, and all in all, the lesions had cleared. (A, c, o, b, ati, P, xpx)\newline%
The completion of the study showed that in clinical trials, people who had been diagnosed with the virus were shown to live longer than those who were not infected and also had “good health.”\newline%
Researchers and clinicians “dedicated to this finding are encouraging,” said Dr. Richard Freidweiler, president of the American Heart Association. “This study is a breakthrough in our efforts to battle a potentially life{-}threatening genetic disease.”\newline%

%


\begin{figure}[h!]%
\centering%
\includegraphics[width=120px]{./photos_from_epoch_8/samples_8_316.png}%
\caption{a man in a suit and tie holding a camera .}%
\end{figure}

%
\end{document}