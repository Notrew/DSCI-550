\documentclass{article}%
\usepackage[T1]{fontenc}%
\usepackage[utf8]{inputenc}%
\usepackage{lmodern}%
\usepackage{textcomp}%
\usepackage{lastpage}%
\usepackage{graphicx}%
%
\title{8\_ jliu\_mdanderson\_org\_NIH Public AccessAuthor ManuscriptInt}%
\author{\textit{Ts'ui Xiuxiu}}%
\date{08-22-1997}%
%
\begin{document}%
\normalsize%
\maketitle%
\section{Interesting issue : Whether or not decision makers and government bodies may be unable to read, dig through, or hold onto documents before we submit them to the National Library in Zimbabwe?\newline%
This must be addressed with the recent release of the mystery manuscript of one Robert Edds (Advocate for Inter{-}Power Project)}%
\label{sec:InterestingissueWhetherornotdecisionmakersandgovernmentbodiesmaybeunabletoread,digthrough,orholdontodocumentsbeforewesubmitthemtotheNationalLibraryinZimbabwe?ThismustbeaddressedwiththerecentreleaseofthemysterymanuscriptofoneRobertEdds(AdvocateforInter{-}PowerProject)}%
Interesting issue : Whether or not decision makers and government bodies may be unable to read, dig through, or hold onto documents before we submit them to the National Library in Zimbabwe?\newline%
This must be addressed with the recent release of the mystery manuscript of one Robert Edds (Advocate for Inter{-}Power Project).\newline%
Freed by an NSL committee following a widely reported reprintsing in a cache of copies of his first work, this mystery manuscript was the first official release of a private monograph out of the Open University of Zimbabwe, in 2002 and is the first public publication of a private monograph out of the Open University of Zimbabwe!\newline%
Why is this an important issue for a public access professional in Zimbabwe?\newline%
An article in last year’s Journal of Access Opportunities and his Private monograph in 1990; When Acquaintances Seek (OPWR) brings the exciting issue to light. The response from researchers at the National Library (Library) for release of these completed manuscripts has been remarkable. To date at least 27 independent researchers have collaborated on the manuscripts in the Library. These are carefully considered manuscripts intended for the public to study and read before submission, and some journals, are also providing private lignum publishing by providing funding for the work.\newline%
Several leading researchers have been contacted by both scholars and the public sector in Zimbabwe. The most recent was Dr X. Izuni, an OYO project manager at the OYO International Research Centre for Africa. In a statement issued in May this year, the OYO Programme Coordinator, Dr Izuni stated that “in the wake of the publication of \#1 Edition 207, we had received numerous requests from individuals, organizations and authorities for access to these manuscripts.\newline%
Sharing the initial response from a leading researcher we were informed by a campaign within Open University which gave us the opportunity to print the manuscript in NVIOR clear draft format. The process has been very compelling and the release of the manuscript has confirmed the commitment of the OYO Programme Coordinator to all stakeholders involved.”\newline%
Dr Izuni further emphasized that we have been discussing this issue since year{-}end as we extend our support and confidence for the grant of a copy of this manuscript by one of our fellow indigenes at Open University in Zimbabwe. We are also working closely with the PF (National Center for Access and Surplus Printers) and the TRC (University of Zimbabwe Network and National Library of Zimbabwe).\newline%
The publication of the manuscript was only completed in February this year, and the selection was due to add value to the project. The approval process started soon after the publication of the manuscript in NVIOR clear draft format, and began at the Open University in February this year.\newline%
The Legal guidelines for offering private monographs or pending manuscripts cannot be reviewed by the Public Authorisation Regulatory Network (PAN) in Zimbabwe, but we hope to be given access to the manuscript by the people we contact. We would like to thank the Irish and Zimbabwean indigenes for their contributions in publishing this work.\newline%

%


\begin{figure}[h!]%
\centering%
\includegraphics[width=120px]{./photos_from_epoch_8/samples_8_239.png}%
\caption{a man wearing a tie and a hat .}%
\end{figure}

%
\end{document}