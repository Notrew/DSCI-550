\documentclass{article}%
\usepackage[T1]{fontenc}%
\usepackage[utf8]{inputenc}%
\usepackage{lmodern}%
\usepackage{textcomp}%
\usepackage{lastpage}%
\usepackage{graphicx}%
%
\title{roductionRecent works in stem cell biology have opened up ne}%
\author{\textit{T'ao Hua}}%
\date{02-21-2000}%
%
\begin{document}%
\normalsize%
\maketitle%
\section{One of the fascinating innovations to biotechnology over the last few years has been the initiation of gene therapy}%
\label{sec:Oneofthefascinatinginnovationstobiotechnologyoverthelastfewyearshasbeentheinitiationofgenetherapy}%
One of the fascinating innovations to biotechnology over the last few years has been the initiation of gene therapy. This involves providing cells of a variety of organs with a novel type of tissue in exchange for energy and nutrients. This method is rapidly becoming used as a global strategy for regenerative medicine, while an even greater potential source of breakthroughs is the contributions of genetics to the treatment of malignant cells.\newline%
Excision cell therapy (or 3G) is currently entering the clinic and has been given widespread recognition for its transformative power in the development of these new therapies. One of the key cell types is sequenced and found to be fed directly to therapeutic neurons. Its development led to new studies showing that Sequenced and ReSeqase cells can translate into more than 90 per cent more potent radiophane and potassium for muscle formation in the placental muscles. More recently, a gene therapy technique that sequenced and found which hybrids could be transplanted out into the human body also appears to be advancing. The technique has also been tested in human stem cells via IV motor neurons.\newline%
According to a Research in Cell Biology (ROCB) report in Psychology (2004), at least 15 embryos have successfully been proscheduled to have a cure for this brain{-}damaging, beleaguered human brain from the usual embryonic stem cells. Furthermore, several individuals have survived surgery to extend their limbs to determine how they had lost their brain stem cells.\newline%
In addition, many will be able to use the approach to treated brain tumors and the kidney to treat bone marrow failure and to resume normal and healthy function in certain areas of the body. This research and trials will enable the European Medicines Agency (EMA) to publish detailed statistics and guidelines regarding the medical and clinical applications of gene therapy.\newline%
The Neutrino, the stem cells derived from a gel derived from chia seeds, is then injected directly into the patient's brain cavity. The cells are created quickly into an environment that spreads through the membrane so that the stem cells can regrow and repair the damaged bone marrow stem cells and restore function to brain stem cells. This system is only possible under rare or severe conditions, such as mesothelioma or some cases cancer. While this research has made the growth of this gene therapy successful, it has also shown that other methods which aim to dissolve a diseased stem cell in a healthy patient need different methods of reducing toxicity. Hence the current technique of targeted biology work can be cited as a powerful tool in achieving more targeted treatments.\newline%

%


\begin{figure}[h!]%
\centering%
\includegraphics[width=120px]{./photos_from_epoch_8/samples_8_222.png}%
\caption{a woman wearing a tie and a hat .}%
\end{figure}

%
\end{document}